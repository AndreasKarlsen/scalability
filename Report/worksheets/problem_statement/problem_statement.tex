\makeatletter \@ifundefined{rootpath}{% Manual to memoir http://mirrors.dotsrc.org/ctan/macros/latex/contrib/memoir/memman.pdf

%\documentclass[a4paper,12pt,fleqn,openany,twoside]{memoir} %two sides for printing
\documentclass[a4paper,12pt,fleqn,openany,oneside]{memoir} %one side for pdf
\usepackage[english]{babel}
\usepackage[utf8]{inputenc}
\usepackage{microtype}
\usepackage{paralist}

%Definitions
\usepackage{amsthm}
\theoremstyle{plain}
\newtheorem{thm}{Theorem}[chapter] % reset theorem numbering for each chapter
\theoremstyle{definition}
\newtheorem{defn}[thm]{Definition}

% Choses the depth of numerations
\setsecnumdepth{subsection}

% Choses the depth of toc
%\maxtocdepth{subsection}

% LaTeX logical statements
\usepackage{ifthen}

% Fancy space after use of e.g. command
\usepackage{xspace}

% Skips after paragraphs
\usepackage{parskip}

% Layout settings
\setlength{\parindent}{0cm}
\setlength{\parskip}{2ex plus 2ex} %kan udvides til f.eks: '2ex plus 2ex minus 0ex'

\sloppybottom

% Don't make a collection per default
\newcommand{\worksheetcollection}{false}

% Bibtex
\usepackage[square,numbers,sort,comma]{natbib}
%\usepackage{cite}
%\bibliographystyle{plainnat}
\bibliographystyle{IEEEtran}


% Fixmes
\usepackage{fixme}
\fxsetup{draft}

% Mathematic
\usepackage{amsmath}
\usepackage{amsfonts}
\usepackage{amssymb}
\usepackage{stmaryrd}
\allowdisplaybreaks[1]


% Acronyms
\usepackage[printonlyused]{acronym}

% Images
\usepackage{graphicx}
\usepackage{wrapfig}
\usepackage[outdir=./]{epstopdf}
\usepackage{epsfig}


% Captions ans subcaptions
\captionnamefont{\footnotesize\bfseries}
\captiontitlefont{\footnotesize}

% Enable memoir subfloats for figures and tables
\newsubfloat{figure}
\newsubfloat{table}

% Hack memoir subfigure styles to have bold label and footnotesize fonts
\renewcommand{\thesubfigure}{\footnotesize\bfseries{(\alph{subfigure})}}
\renewcommand{\thesubtable}{\footnotesize\bfseries{(\alph{subtable})}}

\renewcommand{\subcaption}[2][]{\subbottom[\footnotesize{#1}]{#2}}

% Memoir tweak pagenumbers
%\pagestyle{headings}

% Tikz
\usepackage{tikz}
\usetikzlibrary{arrows,shapes,calc,positioning}
\pgfmathsetseed{1}

%Pgf plots
\usepackage{pgfplots}
\pgfplotsset{compat=1.5}
% loatbarrier, keep figures within (sub,subsub) sections
\usepackage{placeins}
\usepackage{pgfplots}
\usepgfplotslibrary{units}
\usepackage[space-before-unit,range-units = repeat]{siunitx}

% Hyperlinked auto references
\usepackage[hidelinks]{hyperref}
\usepackage[nameinlink]{cleveref}
\crefname{lstlisting}{Listing}{Listings}  
\Crefname{lstlisting}{Listing}{Listings}
%\def\chapterautorefname{Kapitel}
%\def\sectionautorefname{Afsnit}
%\def\subsectionautorefname{Afsnit}
%\def\subsubsectionautorefname{Underafsnit}
%\def\figureautorefname{Figur}
%\def\lstlistingautorefname{Listing}
%\def\lstnumberautorefname{Linje}
%\def\itemautorefname{Punkt}
\usepackage[hypcap]{caption} % Link to top of the figure and not the caption

%Sick shit to make \Autoref command
%http://tex.stackexchange.com/questions/36575/autorefs-inserted-text-has-not-the-correct-case
\def\HyLang@english{%
  \def\equationautorefname{Equation}%
  \def\footnoteautorefname{Footnote}%
  \def\itemautorefname{item}%
  \def\figureautorefname{Figure}%
  \def\tableautorefname{Table}%
  \def\partautorefname{Part}%
  \def\appendixautorefname{Appendix}%
  \def\chapterautorefname{Chapter}%
  \def\sectionautorefname{Section}%
  \def\subsectionautorefname{Subsection}%
  \def\subsubsectionautorefname{Subsubsection}%
  \def\paragraphautorefname{Paragraph}%
  \def\subparagraphautorefname{Subparagraph}%
  \def\FancyVerbLineautorefname{Line}%
  \def\theoremautorefname{Theorem}%
  \def\pageautorefname{Page}%
}

% Reference greencommentssections with number and name
\usepackage{nameref}
\newcommand{\bsnameref}[1]{\Cref{#1} ``\nameref{#1}''}
\newcommand{\bsref}[1]{\Cref{#1}}
\newcommand{\bsbilagref}[1]{Appendix \ref{#1}}
\newcommand{\bsbilagnameref}[1]{Appendix \ref{#1} ``\nameref{#1}''}
\newcommand{\pling}[1]{``#1''}


% Listings for code qoutes
\usepackage{listings}
%\usepackage[usenames,dvipsnames,svgnames,table]{xcolor}
\usepackage{color}
\usepackage{xcolor}
\definecolor{bluekeywords}{rgb}{0.13,0.13,1}
\definecolor{greencomments}{rgb}{0,0.5,0}
\definecolor{redstrings}{rgb}{0.9,0,0}
\usepackage{caption} 
\usepackage{multicol}
\DeclareCaptionFont{white}{\color{white}}
\DeclareCaptionFormat{listing}{\colorbox{gray}{\parbox{\textwidth}{#1#2#3}}}
\captionsetup[lstlisting]{format=listing,labelfont=white,textfont=white}
%\lstset{numbers=left}
\lstset{
	basicstyle=\footnotesize,
	tabsize=2,
	breaklines=true,
  literate={æ}{{\ae}}1 {ø}{{\o}}1 {å}{{\aa}}1 {Æ}{{\AE}}1 {Ø}{{\O}}1 {Å}{{\AA}}1,
  keywords={typeof, new, true, false, catch, function, return, null, catch, switch, var, if, in, while, do, else, case, break},
  keywordstyle=\color{blue}\bfseries,
  ndkeywords={class, export, boolean, throw, implements, import, this},
  ndkeywordstyle=\color{darkgray}\bfseries,
  identifierstyle=\color{black},
  sensitive=false,
  comment=[l]{//},
  morecomment=[s]{/*}{*/},
  commentstyle=\color{purple}\ttfamily,
  stringstyle=\color{red}\ttfamily,
  numbers=left,
  numbersep=-5pt,
  showstringspaces=false,
  showspaces=false,
  %morestring=[b]',
  %morestring=[b]"
}
\lstnewenvironment{code}[1][]%
  {\minipage{\linewidth} 
   \lstset{basicstyle=\ttfamily\footnotesize,frame=single,#1}}
  {\endminipage}


% Worksheet commands
\newcommand{\worksheetstart}[5]{ %Title, Revision, Date, Author, rootpath
	\ifthenelse{\equal{\worksheetcollection}{false}}{
		\newcommand{\rootpath}{#5}
		\documentheader
		\chapter{#1}
	}{
		\chapter{#1}
	}
	\vspace{-1em}
	\textbf{\tiny Revision #2 at #3. Written by #4}\\
	\vspace{2em}\\
}

\newcommand{\worksheetend}{
	\ifthenelse{\equal{\worksheetcollection}{false}}{
		\collectionend
	}{}
}

\newcommand{\documentheader}{
	% Draws a tikz camera
% #1 is the coordinate to the top left corner
% #2 is a label for the righthand center position
% #3 is the text shown in the center of the camera
\newcommand{\camera}[3]{
\coordinate (anchor) at #1;
\draw (anchor) -- ($ (anchor) + (0em,-20pt) $) -- ($ (anchor) + (10pt, -15pt) $) -- ($ (anchor) + (10pt,-5pt)$) -- cycle;
\draw ($ (anchor) + (10pt,-5pt) $) -- ($ (anchor) + (10pt,0pt) $) -- ($ (anchor) + (50pt,0pt) $) -- ($ (anchor) + (50pt,-20pt) $) -- node[yshift=10pt] {\tiny #3} ($ (anchor) + (10pt,-20pt) $)-- cycle;
\coordinate (#2) at ($ (anchor) + (50pt,-10pt) $);
}

\newcounter{frameNumber}
\newcommand{\frameWithSize}[3][false]{
	\stepcounter{frameNumber}
	\coordinate (anchor) at #2;
	\ifthenelse{\equal{#1}{false}}{
		\def\frameNumber{\arabic{frameNumber}}
	}{
		\def\frameNumber{#1}
	}
	\pgfmathtruncatemacro\randomnumber{random(0,4)}
	\node[yshift=20pt] at (anchor) {\frameNumber};
	\ifthenelse{\equal{#3}{I}}{
		\node[draw, minimum size=20pt, fill=green!60] at (anchor) {I};
		\filldraw[fill=gray] ($(anchor) + (-10pt,-40pt)$) rectangle ($(anchor) + (10pt,-20pt) + (0pt,\randomnumber pt)$);
	}{
		\ifthenelse{\equal{#3}{P}}{
			\node[draw, minimum size=20pt, fill=yellow!60] at (anchor) {P};
			\filldraw[fill=gray] ($(anchor) + (-10pt,-40pt)$) rectangle ($(anchor) + (10pt,-30pt) + (0pt,\randomnumber pt)$);
		}{
			\node[draw, minimum size=20pt, fill=blue!40!yellow!60!black] at (anchor) {\color{white}B};
			\filldraw[fill=gray] ($(anchor) + (-10pt,-40pt)$) rectangle ($(anchor) + (10pt,-37pt) + (0pt,\randomnumber pt)$);
		}
	}
	\draw[thick] ($(anchor) + (-10pt,-40pt)$) -- +(20pt,0pt);
}

	\begin{document}
	%\renewcommand{\chaptername}{Worksheet}
	\chapterstyle{section}
	\renewcommand{\beforechapskip}{0pt}
	\renewcommand{\afterchapskip}{0pt}
}

\newcommand{\collectionstart}[1]{
	\newcommand{\rootpath}{#1}
	\renewcommand{\worksheetcollection}{true}
	\documentheader
	\frontmatter
	%\forside
	\makeatletter \@ifundefined{rootpath}{% Manual to memoir http://mirrors.dotsrc.org/ctan/macros/latex/contrib/memoir/memman.pdf

%\documentclass[a4paper,12pt,fleqn,openany,twoside]{memoir} %two sides for printing
\documentclass[a4paper,12pt,fleqn,openany,oneside]{memoir} %one side for pdf
\usepackage[english]{babel}
\usepackage[utf8]{inputenc}
\usepackage{microtype}
\usepackage{paralist}

%Definitions
\usepackage{amsthm}
\theoremstyle{plain}
\newtheorem{thm}{Theorem}[chapter] % reset theorem numbering for each chapter
\theoremstyle{definition}
\newtheorem{defn}[thm]{Definition}

% Choses the depth of numerations
\setsecnumdepth{subsection}

% Choses the depth of toc
%\maxtocdepth{subsection}

% LaTeX logical statements
\usepackage{ifthen}

% Fancy space after use of e.g. command
\usepackage{xspace}

% Skips after paragraphs
\usepackage{parskip}

% Layout settings
\setlength{\parindent}{0cm}
\setlength{\parskip}{2ex plus 2ex} %kan udvides til f.eks: '2ex plus 2ex minus 0ex'

\sloppybottom

% Don't make a collection per default
\newcommand{\worksheetcollection}{false}

% Bibtex
\usepackage[square,numbers,sort,comma]{natbib}
%\usepackage{cite}
%\bibliographystyle{plainnat}
\bibliographystyle{IEEEtran}


% Fixmes
\usepackage{fixme}
\fxsetup{draft}

% Mathematic
\usepackage{amsmath}
\usepackage{amsfonts}
\usepackage{amssymb}
\usepackage{stmaryrd}
\allowdisplaybreaks[1]


% Acronyms
\usepackage[printonlyused]{acronym}

% Images
\usepackage{graphicx}
\usepackage{wrapfig}
\usepackage[outdir=./]{epstopdf}
\usepackage{epsfig}


% Captions ans subcaptions
\captionnamefont{\footnotesize\bfseries}
\captiontitlefont{\footnotesize}

% Enable memoir subfloats for figures and tables
\newsubfloat{figure}
\newsubfloat{table}

% Hack memoir subfigure styles to have bold label and footnotesize fonts
\renewcommand{\thesubfigure}{\footnotesize\bfseries{(\alph{subfigure})}}
\renewcommand{\thesubtable}{\footnotesize\bfseries{(\alph{subtable})}}

\renewcommand{\subcaption}[2][]{\subbottom[\footnotesize{#1}]{#2}}

% Memoir tweak pagenumbers
%\pagestyle{headings}

% Tikz
\usepackage{tikz}
\usetikzlibrary{arrows,shapes,calc,positioning}
\pgfmathsetseed{1}

%Pgf plots
\usepackage{pgfplots}
\pgfplotsset{compat=1.5}
% loatbarrier, keep figures within (sub,subsub) sections
\usepackage{placeins}
\usepackage{pgfplots}
\usepgfplotslibrary{units}
\usepackage[space-before-unit,range-units = repeat]{siunitx}

% Hyperlinked auto references
\usepackage[hidelinks]{hyperref}
\usepackage[nameinlink]{cleveref}
\crefname{lstlisting}{Listing}{Listings}  
\Crefname{lstlisting}{Listing}{Listings}
%\def\chapterautorefname{Kapitel}
%\def\sectionautorefname{Afsnit}
%\def\subsectionautorefname{Afsnit}
%\def\subsubsectionautorefname{Underafsnit}
%\def\figureautorefname{Figur}
%\def\lstlistingautorefname{Listing}
%\def\lstnumberautorefname{Linje}
%\def\itemautorefname{Punkt}
\usepackage[hypcap]{caption} % Link to top of the figure and not the caption

%Sick shit to make \Autoref command
%http://tex.stackexchange.com/questions/36575/autorefs-inserted-text-has-not-the-correct-case
\def\HyLang@english{%
  \def\equationautorefname{Equation}%
  \def\footnoteautorefname{Footnote}%
  \def\itemautorefname{item}%
  \def\figureautorefname{Figure}%
  \def\tableautorefname{Table}%
  \def\partautorefname{Part}%
  \def\appendixautorefname{Appendix}%
  \def\chapterautorefname{Chapter}%
  \def\sectionautorefname{Section}%
  \def\subsectionautorefname{Subsection}%
  \def\subsubsectionautorefname{Subsubsection}%
  \def\paragraphautorefname{Paragraph}%
  \def\subparagraphautorefname{Subparagraph}%
  \def\FancyVerbLineautorefname{Line}%
  \def\theoremautorefname{Theorem}%
  \def\pageautorefname{Page}%
}

% Reference greencommentssections with number and name
\usepackage{nameref}
\newcommand{\bsnameref}[1]{\Cref{#1} ``\nameref{#1}''}
\newcommand{\bsref}[1]{\Cref{#1}}
\newcommand{\bsbilagref}[1]{Appendix \ref{#1}}
\newcommand{\bsbilagnameref}[1]{Appendix \ref{#1} ``\nameref{#1}''}
\newcommand{\pling}[1]{``#1''}


% Listings for code qoutes
\usepackage{listings}
%\usepackage[usenames,dvipsnames,svgnames,table]{xcolor}
\usepackage{color}
\usepackage{xcolor}
\definecolor{bluekeywords}{rgb}{0.13,0.13,1}
\definecolor{greencomments}{rgb}{0,0.5,0}
\definecolor{redstrings}{rgb}{0.9,0,0}
\usepackage{caption} 
\usepackage{multicol}
\DeclareCaptionFont{white}{\color{white}}
\DeclareCaptionFormat{listing}{\colorbox{gray}{\parbox{\textwidth}{#1#2#3}}}
\captionsetup[lstlisting]{format=listing,labelfont=white,textfont=white}
%\lstset{numbers=left}
\lstset{
	basicstyle=\footnotesize,
	tabsize=2,
	breaklines=true,
  literate={æ}{{\ae}}1 {ø}{{\o}}1 {å}{{\aa}}1 {Æ}{{\AE}}1 {Ø}{{\O}}1 {Å}{{\AA}}1,
  keywords={typeof, new, true, false, catch, function, return, null, catch, switch, var, if, in, while, do, else, case, break},
  keywordstyle=\color{blue}\bfseries,
  ndkeywords={class, export, boolean, throw, implements, import, this},
  ndkeywordstyle=\color{darkgray}\bfseries,
  identifierstyle=\color{black},
  sensitive=false,
  comment=[l]{//},
  morecomment=[s]{/*}{*/},
  commentstyle=\color{purple}\ttfamily,
  stringstyle=\color{red}\ttfamily,
  numbers=left,
  numbersep=-5pt,
  showstringspaces=false,
  showspaces=false,
  %morestring=[b]',
  %morestring=[b]"
}
\lstnewenvironment{code}[1][]%
  {\minipage{\linewidth} 
   \lstset{basicstyle=\ttfamily\footnotesize,frame=single,#1}}
  {\endminipage}


% Worksheet commands
\newcommand{\worksheetstart}[5]{ %Title, Revision, Date, Author, rootpath
	\ifthenelse{\equal{\worksheetcollection}{false}}{
		\newcommand{\rootpath}{#5}
		\documentheader
		\chapter{#1}
	}{
		\chapter{#1}
	}
	\vspace{-1em}
	\textbf{\tiny Revision #2 at #3. Written by #4}\\
	\vspace{2em}\\
}

\newcommand{\worksheetend}{
	\ifthenelse{\equal{\worksheetcollection}{false}}{
		\collectionend
	}{}
}

\newcommand{\documentheader}{
	\input{\rootpath/setup/tikz-commands.tex}
	\begin{document}
	%\renewcommand{\chaptername}{Worksheet}
	\chapterstyle{section}
	\renewcommand{\beforechapskip}{0pt}
	\renewcommand{\afterchapskip}{0pt}
}

\newcommand{\collectionstart}[1]{
	\newcommand{\rootpath}{#1}
	\renewcommand{\worksheetcollection}{true}
	\documentheader
	\frontmatter
	%\forside
	\input{\rootpath/worksheets/titlepage/titlepage}
	\input{\rootpath/worksheets/preface/preface}
	%\input{\rootpath/worksheets/forord/forord}
	\newpage
	\newpage
	\tableofcontents*
	\mainmatter
}

\newcommand{\collectionend}{
	\backmatter
	\chapter{List of Acronyms}\vspace{3em}
	\input{\rootpath/setup/acronyms}
	\bibliography{\rootpath/setup/bibliography}
	\end{document}
}


\newcommand{\bscode}{
	\lstinline
}

\newcommand{\bscodemath}[1]{
	\text{\lstinline|#1|}
}

\newcommand{\bsqoute}[2]{
	\begin{quote}
		\textit{``#1''}
		\begin{center}
			-- \emph{#2}
		\end{center}
	\end{quote}
}


\newcommand{\lag}{\langle}
\newcommand{\rag}{\rangle}
\newcommand{\besk}[1]{\ensuremath{\lag #1 \rag}}

\newcommand{\namedtodo}[5]
{
  \ifthenelse{\equal{#1}{}}
  {
    \todo[color=#4,caption=
    {\textbf{#3: } #2}]
    {\color{#5}\textbf{#3: }#2}
  }
  {
    \todo[color=#4,caption=
    {\textbf{#3: } #1}
    ,inline]
    {\color{#5}\textbf{#3: }#2}
  }
}
\newcommand{\andreas}[2][]{\namedtodo{#1}{#2}{Andreas}{blue!50!red!10}{black}}
\newcommand{\lone}[2][]{\namedtodo{#1}{#2}{Lone}{orange}{black}}
\definecolor{babypink}{rgb}{0.96, 0.76, 0.76}
\newcommand{\toby}[2][]{\namedtodo{#1}{#2}{Tobias}{babypink}{black}}
\newcommand{\kasper}[2][]{\namedtodo{#1}{#2}{Kasper}{green}{black}}

%multicol
\usepackage{multicol}

% todonotes
%\usepackage[disable]{todonotes} %For final report
\usepackage{todonotes} %For writing notes
\usepackage{fancyvrb}

%Loading AAU macro
\usepackage{lastpage}
%%%%%%%%%%%%%%%%%%%%%%%%%%%%%%%%%%%%%%%%%%%%%%%%
% Macros for the titlepage
%%%%%%%%%%%%%%%%%%%%%%%%%%%%%%%%%%%%%%%%%%%%%%%%
%Creates the aau titlepage
\newcommand{\aautitlepage}[3]{%
  {
    %set up various length
    \ifx\titlepageleftcolumnwidth\undefined
      \newlength{\titlepageleftcolumnwidth}
      \newlength{\titlepagerightcolumnwidth}
    \fi
    \setlength{\titlepageleftcolumnwidth}{0.5\textwidth-\tabcolsep}
    \setlength{\titlepagerightcolumnwidth}{\textwidth-2\tabcolsep-\titlepageleftcolumnwidth}
    %create title page
    \thispagestyle{empty}
    \noindent%
    \begin{tabular}{@{}ll@{}}
      \parbox{\titlepageleftcolumnwidth}{
        \iflanguage{danish}{%
          \includegraphics[width=\titlepageleftcolumnwidth]{titlepage/figures/aau_logo_da}
        }{%
          \includegraphics[width=\titlepageleftcolumnwidth]{titlepage/figures/aau_logo_en}
        }
      } &
      \parbox{\titlepagerightcolumnwidth}{\raggedleft\small
        #2
      }\bigskip\\
       #1 &
      \parbox[t]{\titlepagerightcolumnwidth}{%
      \textbf{Abstract:}\bigskip\par
        \fbox{\parbox{\titlepagerightcolumnwidth-2\fboxsep-2\fboxrule}{%
          #3
        }}
      }\\
    \end{tabular}
    \vfill  
    \clearpage
  }
}

% Environment for problem statements
% Can be auto referenced.
\newtheorem{problem}{Problem}
\def\problemautorefname{Problem}

%Create english project info
\newcommand{\englishprojectinfo}[6]{%
  \parbox[t]{\titlepageleftcolumnwidth}{
    \textbf{Title:}\\ #1\bigskip\par
    %\textbf{Theme:}\\ #2\bigskip\par
    \textbf{Project Period:}\\ #2\bigskip\par
    \textbf{Project Group:}\\ #3\bigskip\par
    \textbf{Participants:}\\ #4\bigskip\par
    \textbf{Supervisor:}\\ #5\bigskip\par
    %\textbf{Copies:} #6\bigskip\par
    \textbf{Page Numbers:} \pageref{LastPage}\bigskip\par
    \textbf{Date of Completion:}\\ #6
  }
}



%Create danish project info
%\newcommand{\danishprojectinfo}[7]{%
 % \parbox[t]{\titlepageleftcolumnwidth}{
 %   \textbf{Titel:}\\ #1\bigskip\par
%    %\textbf{Tema:}\\ #2\bigskip\par
%    \textbf{Projektperiode:}\\ #2\bigskip\par
%    \textbf{Projektgruppe:}\\ #4\bigskip\par
%    \textbf{Deltager(e):}\\ #5\bigskip\par
 %   \textbf{Vejleder(e):}\\ #6\bigskip\par
%    \textbf{Oplagstal:} #7\bigskip\par
   % \textbf{Sidetal:} \pageref{LastPage}\bigskip\par
  %  \textbf{Afleveringsdato:}\\ #8
 % }
%}
}\makeatother
%\worksheetstart{Titlepage}{0}{December 31, 2012}{../../}
\begin{titlingpage}
\aautitlepage{%
  \englishprojectinfo{
    Investigation into hyped concurrency models %title
  }{%
    Fall Semester 2014 %project period
  }{%
    dpt907e14  % project group
  }{%
    %list of group members
    Tobias Ugleholdt Hansen\\
    Andreas Pørtner Karlsen\\ 
    Kasper Breinholt Laurberg\\
  }{%
    %list of supervisors
     Lone Leth Thomsen
  }{%
    \today % date of completion
  }%
}{%department and address
  \textbf{Department of Computer Science}\\
  Selma Lagerløfs Vej 300\\
  DK-9220 Aalborg Ø\\
  \href{http://www.cs.aau.dk}{http://www.cs.aau.dk}
}{% the abstract
Some nice abstract
}
\end{titlingpage}
	\makeatletter \@ifundefined{rootpath}{% Manual to memoir http://mirrors.dotsrc.org/ctan/macros/latex/contrib/memoir/memman.pdf

%\documentclass[a4paper,12pt,fleqn,openany,twoside]{memoir} %two sides for printing
\documentclass[a4paper,12pt,fleqn,openany,oneside]{memoir} %one side for pdf
\usepackage[english]{babel}
\usepackage[utf8]{inputenc}
\usepackage{microtype}
\usepackage{paralist}

%Definitions
\usepackage{amsthm}
\theoremstyle{plain}
\newtheorem{thm}{Theorem}[chapter] % reset theorem numbering for each chapter
\theoremstyle{definition}
\newtheorem{defn}[thm]{Definition}

% Choses the depth of numerations
\setsecnumdepth{subsection}

% Choses the depth of toc
%\maxtocdepth{subsection}

% LaTeX logical statements
\usepackage{ifthen}

% Fancy space after use of e.g. command
\usepackage{xspace}

% Skips after paragraphs
\usepackage{parskip}

% Layout settings
\setlength{\parindent}{0cm}
\setlength{\parskip}{2ex plus 2ex} %kan udvides til f.eks: '2ex plus 2ex minus 0ex'

\sloppybottom

% Don't make a collection per default
\newcommand{\worksheetcollection}{false}

% Bibtex
\usepackage[square,numbers,sort,comma]{natbib}
%\usepackage{cite}
%\bibliographystyle{plainnat}
\bibliographystyle{IEEEtran}


% Fixmes
\usepackage{fixme}
\fxsetup{draft}

% Mathematic
\usepackage{amsmath}
\usepackage{amsfonts}
\usepackage{amssymb}
\usepackage{stmaryrd}
\allowdisplaybreaks[1]


% Acronyms
\usepackage[printonlyused]{acronym}

% Images
\usepackage{graphicx}
\usepackage{wrapfig}
\usepackage[outdir=./]{epstopdf}
\usepackage{epsfig}


% Captions ans subcaptions
\captionnamefont{\footnotesize\bfseries}
\captiontitlefont{\footnotesize}

% Enable memoir subfloats for figures and tables
\newsubfloat{figure}
\newsubfloat{table}

% Hack memoir subfigure styles to have bold label and footnotesize fonts
\renewcommand{\thesubfigure}{\footnotesize\bfseries{(\alph{subfigure})}}
\renewcommand{\thesubtable}{\footnotesize\bfseries{(\alph{subtable})}}

\renewcommand{\subcaption}[2][]{\subbottom[\footnotesize{#1}]{#2}}

% Memoir tweak pagenumbers
%\pagestyle{headings}

% Tikz
\usepackage{tikz}
\usetikzlibrary{arrows,shapes,calc,positioning}
\pgfmathsetseed{1}

%Pgf plots
\usepackage{pgfplots}
\pgfplotsset{compat=1.5}
% loatbarrier, keep figures within (sub,subsub) sections
\usepackage{placeins}
\usepackage{pgfplots}
\usepgfplotslibrary{units}
\usepackage[space-before-unit,range-units = repeat]{siunitx}

% Hyperlinked auto references
\usepackage[hidelinks]{hyperref}
\usepackage[nameinlink]{cleveref}
\crefname{lstlisting}{Listing}{Listings}  
\Crefname{lstlisting}{Listing}{Listings}
%\def\chapterautorefname{Kapitel}
%\def\sectionautorefname{Afsnit}
%\def\subsectionautorefname{Afsnit}
%\def\subsubsectionautorefname{Underafsnit}
%\def\figureautorefname{Figur}
%\def\lstlistingautorefname{Listing}
%\def\lstnumberautorefname{Linje}
%\def\itemautorefname{Punkt}
\usepackage[hypcap]{caption} % Link to top of the figure and not the caption

%Sick shit to make \Autoref command
%http://tex.stackexchange.com/questions/36575/autorefs-inserted-text-has-not-the-correct-case
\def\HyLang@english{%
  \def\equationautorefname{Equation}%
  \def\footnoteautorefname{Footnote}%
  \def\itemautorefname{item}%
  \def\figureautorefname{Figure}%
  \def\tableautorefname{Table}%
  \def\partautorefname{Part}%
  \def\appendixautorefname{Appendix}%
  \def\chapterautorefname{Chapter}%
  \def\sectionautorefname{Section}%
  \def\subsectionautorefname{Subsection}%
  \def\subsubsectionautorefname{Subsubsection}%
  \def\paragraphautorefname{Paragraph}%
  \def\subparagraphautorefname{Subparagraph}%
  \def\FancyVerbLineautorefname{Line}%
  \def\theoremautorefname{Theorem}%
  \def\pageautorefname{Page}%
}

% Reference greencommentssections with number and name
\usepackage{nameref}
\newcommand{\bsnameref}[1]{\Cref{#1} ``\nameref{#1}''}
\newcommand{\bsref}[1]{\Cref{#1}}
\newcommand{\bsbilagref}[1]{Appendix \ref{#1}}
\newcommand{\bsbilagnameref}[1]{Appendix \ref{#1} ``\nameref{#1}''}
\newcommand{\pling}[1]{``#1''}


% Listings for code qoutes
\usepackage{listings}
%\usepackage[usenames,dvipsnames,svgnames,table]{xcolor}
\usepackage{color}
\usepackage{xcolor}
\definecolor{bluekeywords}{rgb}{0.13,0.13,1}
\definecolor{greencomments}{rgb}{0,0.5,0}
\definecolor{redstrings}{rgb}{0.9,0,0}
\usepackage{caption} 
\usepackage{multicol}
\DeclareCaptionFont{white}{\color{white}}
\DeclareCaptionFormat{listing}{\colorbox{gray}{\parbox{\textwidth}{#1#2#3}}}
\captionsetup[lstlisting]{format=listing,labelfont=white,textfont=white}
%\lstset{numbers=left}
\lstset{
	basicstyle=\footnotesize,
	tabsize=2,
	breaklines=true,
  literate={æ}{{\ae}}1 {ø}{{\o}}1 {å}{{\aa}}1 {Æ}{{\AE}}1 {Ø}{{\O}}1 {Å}{{\AA}}1,
  keywords={typeof, new, true, false, catch, function, return, null, catch, switch, var, if, in, while, do, else, case, break},
  keywordstyle=\color{blue}\bfseries,
  ndkeywords={class, export, boolean, throw, implements, import, this},
  ndkeywordstyle=\color{darkgray}\bfseries,
  identifierstyle=\color{black},
  sensitive=false,
  comment=[l]{//},
  morecomment=[s]{/*}{*/},
  commentstyle=\color{purple}\ttfamily,
  stringstyle=\color{red}\ttfamily,
  numbers=left,
  numbersep=-5pt,
  showstringspaces=false,
  showspaces=false,
  %morestring=[b]',
  %morestring=[b]"
}
\lstnewenvironment{code}[1][]%
  {\minipage{\linewidth} 
   \lstset{basicstyle=\ttfamily\footnotesize,frame=single,#1}}
  {\endminipage}


% Worksheet commands
\newcommand{\worksheetstart}[5]{ %Title, Revision, Date, Author, rootpath
	\ifthenelse{\equal{\worksheetcollection}{false}}{
		\newcommand{\rootpath}{#5}
		\documentheader
		\chapter{#1}
	}{
		\chapter{#1}
	}
	\vspace{-1em}
	\textbf{\tiny Revision #2 at #3. Written by #4}\\
	\vspace{2em}\\
}

\newcommand{\worksheetend}{
	\ifthenelse{\equal{\worksheetcollection}{false}}{
		\collectionend
	}{}
}

\newcommand{\documentheader}{
	\input{\rootpath/setup/tikz-commands.tex}
	\begin{document}
	%\renewcommand{\chaptername}{Worksheet}
	\chapterstyle{section}
	\renewcommand{\beforechapskip}{0pt}
	\renewcommand{\afterchapskip}{0pt}
}

\newcommand{\collectionstart}[1]{
	\newcommand{\rootpath}{#1}
	\renewcommand{\worksheetcollection}{true}
	\documentheader
	\frontmatter
	%\forside
	\input{\rootpath/worksheets/titlepage/titlepage}
	\input{\rootpath/worksheets/preface/preface}
	%\input{\rootpath/worksheets/forord/forord}
	\newpage
	\newpage
	\tableofcontents*
	\mainmatter
}

\newcommand{\collectionend}{
	\backmatter
	\chapter{List of Acronyms}\vspace{3em}
	\input{\rootpath/setup/acronyms}
	\bibliography{\rootpath/setup/bibliography}
	\end{document}
}


\newcommand{\bscode}{
	\lstinline
}

\newcommand{\bscodemath}[1]{
	\text{\lstinline|#1|}
}

\newcommand{\bsqoute}[2]{
	\begin{quote}
		\textit{``#1''}
		\begin{center}
			-- \emph{#2}
		\end{center}
	\end{quote}
}


\newcommand{\lag}{\langle}
\newcommand{\rag}{\rangle}
\newcommand{\besk}[1]{\ensuremath{\lag #1 \rag}}

\newcommand{\namedtodo}[5]
{
  \ifthenelse{\equal{#1}{}}
  {
    \todo[color=#4,caption=
    {\textbf{#3: } #2}]
    {\color{#5}\textbf{#3: }#2}
  }
  {
    \todo[color=#4,caption=
    {\textbf{#3: } #1}
    ,inline]
    {\color{#5}\textbf{#3: }#2}
  }
}
\newcommand{\andreas}[2][]{\namedtodo{#1}{#2}{Andreas}{blue!50!red!10}{black}}
\newcommand{\lone}[2][]{\namedtodo{#1}{#2}{Lone}{orange}{black}}
\definecolor{babypink}{rgb}{0.96, 0.76, 0.76}
\newcommand{\toby}[2][]{\namedtodo{#1}{#2}{Tobias}{babypink}{black}}
\newcommand{\kasper}[2][]{\namedtodo{#1}{#2}{Kasper}{green}{black}}

%multicol
\usepackage{multicol}

% todonotes
%\usepackage[disable]{todonotes} %For final report
\usepackage{todonotes} %For writing notes
\usepackage{fancyvrb}

%Loading AAU macro
\usepackage{lastpage}
%%%%%%%%%%%%%%%%%%%%%%%%%%%%%%%%%%%%%%%%%%%%%%%%
% Macros for the titlepage
%%%%%%%%%%%%%%%%%%%%%%%%%%%%%%%%%%%%%%%%%%%%%%%%
%Creates the aau titlepage
\newcommand{\aautitlepage}[3]{%
  {
    %set up various length
    \ifx\titlepageleftcolumnwidth\undefined
      \newlength{\titlepageleftcolumnwidth}
      \newlength{\titlepagerightcolumnwidth}
    \fi
    \setlength{\titlepageleftcolumnwidth}{0.5\textwidth-\tabcolsep}
    \setlength{\titlepagerightcolumnwidth}{\textwidth-2\tabcolsep-\titlepageleftcolumnwidth}
    %create title page
    \thispagestyle{empty}
    \noindent%
    \begin{tabular}{@{}ll@{}}
      \parbox{\titlepageleftcolumnwidth}{
        \iflanguage{danish}{%
          \includegraphics[width=\titlepageleftcolumnwidth]{titlepage/figures/aau_logo_da}
        }{%
          \includegraphics[width=\titlepageleftcolumnwidth]{titlepage/figures/aau_logo_en}
        }
      } &
      \parbox{\titlepagerightcolumnwidth}{\raggedleft\small
        #2
      }\bigskip\\
       #1 &
      \parbox[t]{\titlepagerightcolumnwidth}{%
      \textbf{Abstract:}\bigskip\par
        \fbox{\parbox{\titlepagerightcolumnwidth-2\fboxsep-2\fboxrule}{%
          #3
        }}
      }\\
    \end{tabular}
    \vfill  
    \clearpage
  }
}

% Environment for problem statements
% Can be auto referenced.
\newtheorem{problem}{Problem}
\def\problemautorefname{Problem}

%Create english project info
\newcommand{\englishprojectinfo}[6]{%
  \parbox[t]{\titlepageleftcolumnwidth}{
    \textbf{Title:}\\ #1\bigskip\par
    %\textbf{Theme:}\\ #2\bigskip\par
    \textbf{Project Period:}\\ #2\bigskip\par
    \textbf{Project Group:}\\ #3\bigskip\par
    \textbf{Participants:}\\ #4\bigskip\par
    \textbf{Supervisor:}\\ #5\bigskip\par
    %\textbf{Copies:} #6\bigskip\par
    \textbf{Page Numbers:} \pageref{LastPage}\bigskip\par
    \textbf{Date of Completion:}\\ #6
  }
}



%Create danish project info
%\newcommand{\danishprojectinfo}[7]{%
 % \parbox[t]{\titlepageleftcolumnwidth}{
 %   \textbf{Titel:}\\ #1\bigskip\par
%    %\textbf{Tema:}\\ #2\bigskip\par
%    \textbf{Projektperiode:}\\ #2\bigskip\par
%    \textbf{Projektgruppe:}\\ #4\bigskip\par
%    \textbf{Deltager(e):}\\ #5\bigskip\par
 %   \textbf{Vejleder(e):}\\ #6\bigskip\par
%    \textbf{Oplagstal:} #7\bigskip\par
   % \textbf{Sidetal:} \pageref{LastPage}\bigskip\par
  %  \textbf{Afleveringsdato:}\\ #8
 % }
%}
}\makeatother
\worksheetstart{Preface}{1}{December 17, 2013}{Mino}{../../}
This report documents project work done by group dpt907e14 at the Department of Computer Science at Aalborg University. The report was written as part of the Computer Science (IT) study program in the fall of 2014 at the 9th semester.

The first time an acronym is used it will appear in the format: Threads \& Locks (TL). Inline quotations and names will appear in \textit{italics}. The work presented in this report is based on work or results described in books, articles, video lectures and research papers from outside sources. The full list of acronyms along with the bibliography and appendix, can be found at the end of the report.

One chapter contains a number of mathematical formulas. The notation and intuition of these formulas will be defined after the formula. The exact understanding of the formulas is not important, it is the understanding of the amount of work the formulas will produce when calculated on a computer.

The report was written under the assumption, that the reader has solid understanding of computer science, and basic knowledge of concurrency. The basics of concurrency will be explained in \bsref{chp:con_basics}, and can be skipped if the reader already know this. The report can be read from end to end as a coherent piece, or as the reading guide below suggests.

Chapter 4, 5, and 6 can be read individually to obtain knowledge about a specific concurrency model. Chapter 7 compares the characteristics of the concurrency models and can be read to gain an overview of its characteristic differences. Chapter 8 can be read individually to see the performance result between the models. Chapter 9 can be used as a guide for choosing concurrency model, and contains only a summary of the most important factors.

We would like to give a special thanks to our supervisor Lone Leth Thomsen, from the Department of Computer Science, for excellent guidance, immaculate attention to detail. She supplied invaluable help throughout the project with professionalism and black humour. Her constructive criticism helped us to narrow down the subject and kept us motivated and enthusiastic about the project.


\newpage
\vspace*{30 mm}
%\vspace*{\fill}
\begin{vplace}

\begin{minipage}[b]{0.45\textwidth}
 \centering
 \rule{\textwidth}{0.5pt}\\
  Tobias Ugleholdt Hansen\\
 {\footnotesize tuha13@student.aau.dk}
\end{minipage}
\begin{minipage}[b]{0.45\textwidth}
 \centering
 \rule{\textwidth}{0.5pt}\\
  Andreas Pørtner Karlsen\\
 {\footnotesize akarls13@student.aau.dk}
\end{minipage}\\\\
\begin{minipage}[b]{0.45\textwidth}
 \centering
 \rule{\textwidth}{0.5pt}\\
  Kasper Breinholt Laurberg\\
 {\footnotesize klaurb13@student.aau.dk}
\end{minipage}\\\\


\end{vplace}
\worksheetend

	%\input{\rootpath/worksheets/forord/forord}
	\newpage
	\newpage
	\tableofcontents*
	\mainmatter
}

\newcommand{\collectionend}{
	\backmatter
	\chapter{List of Acronyms}\vspace{3em}
	\begin{acronym}[ITU-T]
\acro{WS}{Webservice}
\end{acronym}

	\bibliography{\rootpath/setup/bibliography}
	\end{document}
}


\newcommand{\bscode}{
	\lstinline
}

\newcommand{\bscodemath}[1]{
	\text{\lstinline|#1|}
}

\newcommand{\bsqoute}[2]{
	\begin{quote}
		\textit{``#1''}
		\begin{center}
			-- \emph{#2}
		\end{center}
	\end{quote}
}


\newcommand{\lag}{\langle}
\newcommand{\rag}{\rangle}
\newcommand{\besk}[1]{\ensuremath{\lag #1 \rag}}

\newcommand{\namedtodo}[5]
{
  \ifthenelse{\equal{#1}{}}
  {
    \todo[color=#4,caption=
    {\textbf{#3: } #2}]
    {\color{#5}\textbf{#3: }#2}
  }
  {
    \todo[color=#4,caption=
    {\textbf{#3: } #1}
    ,inline]
    {\color{#5}\textbf{#3: }#2}
  }
}
\newcommand{\andreas}[2][]{\namedtodo{#1}{#2}{Andreas}{blue!50!red!10}{black}}
\newcommand{\lone}[2][]{\namedtodo{#1}{#2}{Lone}{orange}{black}}
\definecolor{babypink}{rgb}{0.96, 0.76, 0.76}
\newcommand{\toby}[2][]{\namedtodo{#1}{#2}{Tobias}{babypink}{black}}
\newcommand{\kasper}[2][]{\namedtodo{#1}{#2}{Kasper}{green}{black}}

%multicol
\usepackage{multicol}

% todonotes
%\usepackage[disable]{todonotes} %For final report
\usepackage{todonotes} %For writing notes
\usepackage{fancyvrb}

%Loading AAU macro
\usepackage{lastpage}
%%%%%%%%%%%%%%%%%%%%%%%%%%%%%%%%%%%%%%%%%%%%%%%%
% Macros for the titlepage
%%%%%%%%%%%%%%%%%%%%%%%%%%%%%%%%%%%%%%%%%%%%%%%%
%Creates the aau titlepage
\newcommand{\aautitlepage}[3]{%
  {
    %set up various length
    \ifx\titlepageleftcolumnwidth\undefined
      \newlength{\titlepageleftcolumnwidth}
      \newlength{\titlepagerightcolumnwidth}
    \fi
    \setlength{\titlepageleftcolumnwidth}{0.5\textwidth-\tabcolsep}
    \setlength{\titlepagerightcolumnwidth}{\textwidth-2\tabcolsep-\titlepageleftcolumnwidth}
    %create title page
    \thispagestyle{empty}
    \noindent%
    \begin{tabular}{@{}ll@{}}
      \parbox{\titlepageleftcolumnwidth}{
        \iflanguage{danish}{%
          \includegraphics[width=\titlepageleftcolumnwidth]{titlepage/figures/aau_logo_da}
        }{%
          \includegraphics[width=\titlepageleftcolumnwidth]{titlepage/figures/aau_logo_en}
        }
      } &
      \parbox{\titlepagerightcolumnwidth}{\raggedleft\small
        #2
      }\bigskip\\
       #1 &
      \parbox[t]{\titlepagerightcolumnwidth}{%
      \textbf{Abstract:}\bigskip\par
        \fbox{\parbox{\titlepagerightcolumnwidth-2\fboxsep-2\fboxrule}{%
          #3
        }}
      }\\
    \end{tabular}
    \vfill  
    \clearpage
  }
}

% Environment for problem statements
% Can be auto referenced.
\newtheorem{problem}{Problem}
\def\problemautorefname{Problem}

%Create english project info
\newcommand{\englishprojectinfo}[6]{%
  \parbox[t]{\titlepageleftcolumnwidth}{
    \textbf{Title:}\\ #1\bigskip\par
    %\textbf{Theme:}\\ #2\bigskip\par
    \textbf{Project Period:}\\ #2\bigskip\par
    \textbf{Project Group:}\\ #3\bigskip\par
    \textbf{Participants:}\\ #4\bigskip\par
    \textbf{Supervisor:}\\ #5\bigskip\par
    %\textbf{Copies:} #6\bigskip\par
    \textbf{Page Numbers:} \pageref{LastPage}\bigskip\par
    \textbf{Date of Completion:}\\ #6
  }
}



%Create danish project info
%\newcommand{\danishprojectinfo}[7]{%
 % \parbox[t]{\titlepageleftcolumnwidth}{
 %   \textbf{Titel:}\\ #1\bigskip\par
%    %\textbf{Tema:}\\ #2\bigskip\par
%    \textbf{Projektperiode:}\\ #2\bigskip\par
%    \textbf{Projektgruppe:}\\ #4\bigskip\par
%    \textbf{Deltager(e):}\\ #5\bigskip\par
 %   \textbf{Vejleder(e):}\\ #6\bigskip\par
%    \textbf{Oplagstal:} #7\bigskip\par
   % \textbf{Sidetal:} \pageref{LastPage}\bigskip\par
  %  \textbf{Afleveringsdato:}\\ #8
 % }
%}
}\makeatother
\worksheetstart{Problem Statement}{1}{22. September, 2014}{Andreas}{../../}
%
In order to structure our investigation, we have extracted five\toby{Make sure this is the right number in the end} questions. The questions are based on problems found in our preliminary investigation, which is described in the \bsref{chap:intro}. These questions will serve as a guideline our further investigation, and will be used to conclude upon the project in \bsref{chap:conclusion}.

The questions we search to answer are:
\begin{enumerate}
\item Which problems exists with concurrency in a multithreaded environment? 
\item What are the approaches for handling concurrency?
\item In what problem domains are the approaches most applicable?
\item How is the approaches performance comparatively?
\item How can the approaches be applied to a distributed environment?
\end{enumerate}

In our preliminary investigation, we discovered the traditional threads and locks approach, and the more modern actor model, event-driven, and software transactional memory approaches. Threads and locks have been selected based on its widespread use and historical significance, while the remaining models have been chosen based on their more modern and widely different approaches to concurrency. It was deemed that these remaining models provide a broad enough view of what is currently state of the art concurrency models. 

\worksheetend