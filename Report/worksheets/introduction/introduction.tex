\makeatletter \@ifundefined{rootpath}{% Manual to memoir http://mirrors.dotsrc.org/ctan/macros/latex/contrib/memoir/memman.pdf

%\documentclass[a4paper,12pt,fleqn,openany,twoside]{memoir} %two sides for printing
\documentclass[a4paper,12pt,fleqn,openany,oneside]{memoir} %one side for pdf
\usepackage[english]{babel}
\usepackage[utf8]{inputenc}
\usepackage{microtype}
\usepackage{paralist}

%Definitions
\usepackage{amsthm}
\theoremstyle{plain}
\newtheorem{thm}{Theorem}[chapter] % reset theorem numbering for each chapter
\theoremstyle{definition}
\newtheorem{defn}[thm]{Definition}

% Choses the depth of numerations
\setsecnumdepth{subsection}

% Choses the depth of toc
%\maxtocdepth{subsection}

% LaTeX logical statements
\usepackage{ifthen}

% Fancy space after use of e.g. command
\usepackage{xspace}

% Skips after paragraphs
\usepackage{parskip}

% Layout settings
\setlength{\parindent}{0cm}
\setlength{\parskip}{2ex plus 2ex} %kan udvides til f.eks: '2ex plus 2ex minus 0ex'

\sloppybottom

% Don't make a collection per default
\newcommand{\worksheetcollection}{false}

% Bibtex
\usepackage[square,numbers,sort,comma]{natbib}
%\usepackage{cite}
%\bibliographystyle{plainnat}
\bibliographystyle{IEEEtran}


% Fixmes
\usepackage{fixme}
\fxsetup{draft}

% Mathematic
\usepackage{amsmath}
\usepackage{amsfonts}
\usepackage{amssymb}
\usepackage{stmaryrd}
\allowdisplaybreaks[1]


% Acronyms
\usepackage[printonlyused]{acronym}

% Images
\usepackage{graphicx}
\usepackage{wrapfig}
\usepackage[outdir=./]{epstopdf}
\usepackage{epsfig}


% Captions ans subcaptions
\captionnamefont{\footnotesize\bfseries}
\captiontitlefont{\footnotesize}

% Enable memoir subfloats for figures and tables
\newsubfloat{figure}
\newsubfloat{table}

% Hack memoir subfigure styles to have bold label and footnotesize fonts
\renewcommand{\thesubfigure}{\footnotesize\bfseries{(\alph{subfigure})}}
\renewcommand{\thesubtable}{\footnotesize\bfseries{(\alph{subtable})}}

\renewcommand{\subcaption}[2][]{\subbottom[\footnotesize{#1}]{#2}}

% Memoir tweak pagenumbers
%\pagestyle{headings}

% Tikz
\usepackage{tikz}
\usetikzlibrary{arrows,shapes,calc,positioning}
\pgfmathsetseed{1}

%Pgf plots
\usepackage{pgfplots}
\pgfplotsset{compat=1.5}
% loatbarrier, keep figures within (sub,subsub) sections
\usepackage{placeins}
\usepackage{pgfplots}
\usepgfplotslibrary{units}
\usepackage[space-before-unit,range-units = repeat]{siunitx}

% Hyperlinked auto references
\usepackage[hidelinks]{hyperref}
\usepackage[nameinlink]{cleveref}
\crefname{lstlisting}{Listing}{Listings}  
\Crefname{lstlisting}{Listing}{Listings}
%\def\chapterautorefname{Kapitel}
%\def\sectionautorefname{Afsnit}
%\def\subsectionautorefname{Afsnit}
%\def\subsubsectionautorefname{Underafsnit}
%\def\figureautorefname{Figur}
%\def\lstlistingautorefname{Listing}
%\def\lstnumberautorefname{Linje}
%\def\itemautorefname{Punkt}
\usepackage[hypcap]{caption} % Link to top of the figure and not the caption

%Sick shit to make \Autoref command
%http://tex.stackexchange.com/questions/36575/autorefs-inserted-text-has-not-the-correct-case
\def\HyLang@english{%
  \def\equationautorefname{Equation}%
  \def\footnoteautorefname{Footnote}%
  \def\itemautorefname{item}%
  \def\figureautorefname{Figure}%
  \def\tableautorefname{Table}%
  \def\partautorefname{Part}%
  \def\appendixautorefname{Appendix}%
  \def\chapterautorefname{Chapter}%
  \def\sectionautorefname{Section}%
  \def\subsectionautorefname{Subsection}%
  \def\subsubsectionautorefname{Subsubsection}%
  \def\paragraphautorefname{Paragraph}%
  \def\subparagraphautorefname{Subparagraph}%
  \def\FancyVerbLineautorefname{Line}%
  \def\theoremautorefname{Theorem}%
  \def\pageautorefname{Page}%
}

% Reference greencommentssections with number and name
\usepackage{nameref}
\newcommand{\bsnameref}[1]{\Cref{#1} ``\nameref{#1}''}
\newcommand{\bsref}[1]{\Cref{#1}}
\newcommand{\bsbilagref}[1]{Appendix \ref{#1}}
\newcommand{\bsbilagnameref}[1]{Appendix \ref{#1} ``\nameref{#1}''}
\newcommand{\pling}[1]{``#1''}


% Listings for code qoutes
\usepackage{listings}
%\usepackage[usenames,dvipsnames,svgnames,table]{xcolor}
\usepackage{color}
\usepackage{xcolor}
\definecolor{bluekeywords}{rgb}{0.13,0.13,1}
\definecolor{greencomments}{rgb}{0,0.5,0}
\definecolor{redstrings}{rgb}{0.9,0,0}
\usepackage{caption} 
\usepackage{multicol}
\DeclareCaptionFont{white}{\color{white}}
\DeclareCaptionFormat{listing}{\colorbox{gray}{\parbox{\textwidth}{#1#2#3}}}
\captionsetup[lstlisting]{format=listing,labelfont=white,textfont=white}
%\lstset{numbers=left}
\lstset{
	basicstyle=\footnotesize,
	tabsize=2,
	breaklines=true,
  literate={æ}{{\ae}}1 {ø}{{\o}}1 {å}{{\aa}}1 {Æ}{{\AE}}1 {Ø}{{\O}}1 {Å}{{\AA}}1,
  keywords={typeof, new, true, false, catch, function, return, null, catch, switch, var, if, in, while, do, else, case, break},
  keywordstyle=\color{blue}\bfseries,
  ndkeywords={class, export, boolean, throw, implements, import, this},
  ndkeywordstyle=\color{darkgray}\bfseries,
  identifierstyle=\color{black},
  sensitive=false,
  comment=[l]{//},
  morecomment=[s]{/*}{*/},
  commentstyle=\color{purple}\ttfamily,
  stringstyle=\color{red}\ttfamily,
  numbers=left,
  numbersep=-5pt,
  showstringspaces=false,
  showspaces=false,
  %morestring=[b]',
  %morestring=[b]"
}
\lstnewenvironment{code}[1][]%
  {\minipage{\linewidth} 
   \lstset{basicstyle=\ttfamily\footnotesize,frame=single,#1}}
  {\endminipage}


% Worksheet commands
\newcommand{\worksheetstart}[5]{ %Title, Revision, Date, Author, rootpath
	\ifthenelse{\equal{\worksheetcollection}{false}}{
		\newcommand{\rootpath}{#5}
		\documentheader
		\chapter{#1}
	}{
		\chapter{#1}
	}
	\vspace{-1em}
	\textbf{\tiny Revision #2 at #3. Written by #4}\\
	\vspace{2em}\\
}

\newcommand{\worksheetend}{
	\ifthenelse{\equal{\worksheetcollection}{false}}{
		\collectionend
	}{}
}

\newcommand{\documentheader}{
	% Draws a tikz camera
% #1 is the coordinate to the top left corner
% #2 is a label for the righthand center position
% #3 is the text shown in the center of the camera
\newcommand{\camera}[3]{
\coordinate (anchor) at #1;
\draw (anchor) -- ($ (anchor) + (0em,-20pt) $) -- ($ (anchor) + (10pt, -15pt) $) -- ($ (anchor) + (10pt,-5pt)$) -- cycle;
\draw ($ (anchor) + (10pt,-5pt) $) -- ($ (anchor) + (10pt,0pt) $) -- ($ (anchor) + (50pt,0pt) $) -- ($ (anchor) + (50pt,-20pt) $) -- node[yshift=10pt] {\tiny #3} ($ (anchor) + (10pt,-20pt) $)-- cycle;
\coordinate (#2) at ($ (anchor) + (50pt,-10pt) $);
}

\newcounter{frameNumber}
\newcommand{\frameWithSize}[3][false]{
	\stepcounter{frameNumber}
	\coordinate (anchor) at #2;
	\ifthenelse{\equal{#1}{false}}{
		\def\frameNumber{\arabic{frameNumber}}
	}{
		\def\frameNumber{#1}
	}
	\pgfmathtruncatemacro\randomnumber{random(0,4)}
	\node[yshift=20pt] at (anchor) {\frameNumber};
	\ifthenelse{\equal{#3}{I}}{
		\node[draw, minimum size=20pt, fill=green!60] at (anchor) {I};
		\filldraw[fill=gray] ($(anchor) + (-10pt,-40pt)$) rectangle ($(anchor) + (10pt,-20pt) + (0pt,\randomnumber pt)$);
	}{
		\ifthenelse{\equal{#3}{P}}{
			\node[draw, minimum size=20pt, fill=yellow!60] at (anchor) {P};
			\filldraw[fill=gray] ($(anchor) + (-10pt,-40pt)$) rectangle ($(anchor) + (10pt,-30pt) + (0pt,\randomnumber pt)$);
		}{
			\node[draw, minimum size=20pt, fill=blue!40!yellow!60!black] at (anchor) {\color{white}B};
			\filldraw[fill=gray] ($(anchor) + (-10pt,-40pt)$) rectangle ($(anchor) + (10pt,-37pt) + (0pt,\randomnumber pt)$);
		}
	}
	\draw[thick] ($(anchor) + (-10pt,-40pt)$) -- +(20pt,0pt);
}

	\begin{document}
	%\renewcommand{\chaptername}{Worksheet}
	\chapterstyle{section}
	\renewcommand{\beforechapskip}{0pt}
	\renewcommand{\afterchapskip}{0pt}
}

\newcommand{\collectionstart}[1]{
	\newcommand{\rootpath}{#1}
	\renewcommand{\worksheetcollection}{true}
	\documentheader
	\frontmatter
	%\forside
	\makeatletter \@ifundefined{rootpath}{% Manual to memoir http://mirrors.dotsrc.org/ctan/macros/latex/contrib/memoir/memman.pdf

%\documentclass[a4paper,12pt,fleqn,openany,twoside]{memoir} %two sides for printing
\documentclass[a4paper,12pt,fleqn,openany,oneside]{memoir} %one side for pdf
\usepackage[english]{babel}
\usepackage[utf8]{inputenc}
\usepackage{microtype}
\usepackage{paralist}

%Definitions
\usepackage{amsthm}
\theoremstyle{plain}
\newtheorem{thm}{Theorem}[chapter] % reset theorem numbering for each chapter
\theoremstyle{definition}
\newtheorem{defn}[thm]{Definition}

% Choses the depth of numerations
\setsecnumdepth{subsection}

% Choses the depth of toc
%\maxtocdepth{subsection}

% LaTeX logical statements
\usepackage{ifthen}

% Fancy space after use of e.g. command
\usepackage{xspace}

% Skips after paragraphs
\usepackage{parskip}

% Layout settings
\setlength{\parindent}{0cm}
\setlength{\parskip}{2ex plus 2ex} %kan udvides til f.eks: '2ex plus 2ex minus 0ex'

\sloppybottom

% Don't make a collection per default
\newcommand{\worksheetcollection}{false}

% Bibtex
\usepackage[square,numbers,sort,comma]{natbib}
%\usepackage{cite}
%\bibliographystyle{plainnat}
\bibliographystyle{IEEEtran}


% Fixmes
\usepackage{fixme}
\fxsetup{draft}

% Mathematic
\usepackage{amsmath}
\usepackage{amsfonts}
\usepackage{amssymb}
\usepackage{stmaryrd}
\allowdisplaybreaks[1]


% Acronyms
\usepackage[printonlyused]{acronym}

% Images
\usepackage{graphicx}
\usepackage{wrapfig}
\usepackage[outdir=./]{epstopdf}
\usepackage{epsfig}


% Captions ans subcaptions
\captionnamefont{\footnotesize\bfseries}
\captiontitlefont{\footnotesize}

% Enable memoir subfloats for figures and tables
\newsubfloat{figure}
\newsubfloat{table}

% Hack memoir subfigure styles to have bold label and footnotesize fonts
\renewcommand{\thesubfigure}{\footnotesize\bfseries{(\alph{subfigure})}}
\renewcommand{\thesubtable}{\footnotesize\bfseries{(\alph{subtable})}}

\renewcommand{\subcaption}[2][]{\subbottom[\footnotesize{#1}]{#2}}

% Memoir tweak pagenumbers
%\pagestyle{headings}

% Tikz
\usepackage{tikz}
\usetikzlibrary{arrows,shapes,calc,positioning}
\pgfmathsetseed{1}

%Pgf plots
\usepackage{pgfplots}
\pgfplotsset{compat=1.5}
% loatbarrier, keep figures within (sub,subsub) sections
\usepackage{placeins}
\usepackage{pgfplots}
\usepgfplotslibrary{units}
\usepackage[space-before-unit,range-units = repeat]{siunitx}

% Hyperlinked auto references
\usepackage[hidelinks]{hyperref}
\usepackage[nameinlink]{cleveref}
\crefname{lstlisting}{Listing}{Listings}  
\Crefname{lstlisting}{Listing}{Listings}
%\def\chapterautorefname{Kapitel}
%\def\sectionautorefname{Afsnit}
%\def\subsectionautorefname{Afsnit}
%\def\subsubsectionautorefname{Underafsnit}
%\def\figureautorefname{Figur}
%\def\lstlistingautorefname{Listing}
%\def\lstnumberautorefname{Linje}
%\def\itemautorefname{Punkt}
\usepackage[hypcap]{caption} % Link to top of the figure and not the caption

%Sick shit to make \Autoref command
%http://tex.stackexchange.com/questions/36575/autorefs-inserted-text-has-not-the-correct-case
\def\HyLang@english{%
  \def\equationautorefname{Equation}%
  \def\footnoteautorefname{Footnote}%
  \def\itemautorefname{item}%
  \def\figureautorefname{Figure}%
  \def\tableautorefname{Table}%
  \def\partautorefname{Part}%
  \def\appendixautorefname{Appendix}%
  \def\chapterautorefname{Chapter}%
  \def\sectionautorefname{Section}%
  \def\subsectionautorefname{Subsection}%
  \def\subsubsectionautorefname{Subsubsection}%
  \def\paragraphautorefname{Paragraph}%
  \def\subparagraphautorefname{Subparagraph}%
  \def\FancyVerbLineautorefname{Line}%
  \def\theoremautorefname{Theorem}%
  \def\pageautorefname{Page}%
}

% Reference greencommentssections with number and name
\usepackage{nameref}
\newcommand{\bsnameref}[1]{\Cref{#1} ``\nameref{#1}''}
\newcommand{\bsref}[1]{\Cref{#1}}
\newcommand{\bsbilagref}[1]{Appendix \ref{#1}}
\newcommand{\bsbilagnameref}[1]{Appendix \ref{#1} ``\nameref{#1}''}
\newcommand{\pling}[1]{``#1''}


% Listings for code qoutes
\usepackage{listings}
%\usepackage[usenames,dvipsnames,svgnames,table]{xcolor}
\usepackage{color}
\usepackage{xcolor}
\definecolor{bluekeywords}{rgb}{0.13,0.13,1}
\definecolor{greencomments}{rgb}{0,0.5,0}
\definecolor{redstrings}{rgb}{0.9,0,0}
\usepackage{caption} 
\usepackage{multicol}
\DeclareCaptionFont{white}{\color{white}}
\DeclareCaptionFormat{listing}{\colorbox{gray}{\parbox{\textwidth}{#1#2#3}}}
\captionsetup[lstlisting]{format=listing,labelfont=white,textfont=white}
%\lstset{numbers=left}
\lstset{
	basicstyle=\footnotesize,
	tabsize=2,
	breaklines=true,
  literate={æ}{{\ae}}1 {ø}{{\o}}1 {å}{{\aa}}1 {Æ}{{\AE}}1 {Ø}{{\O}}1 {Å}{{\AA}}1,
  keywords={typeof, new, true, false, catch, function, return, null, catch, switch, var, if, in, while, do, else, case, break},
  keywordstyle=\color{blue}\bfseries,
  ndkeywords={class, export, boolean, throw, implements, import, this},
  ndkeywordstyle=\color{darkgray}\bfseries,
  identifierstyle=\color{black},
  sensitive=false,
  comment=[l]{//},
  morecomment=[s]{/*}{*/},
  commentstyle=\color{purple}\ttfamily,
  stringstyle=\color{red}\ttfamily,
  numbers=left,
  numbersep=-5pt,
  showstringspaces=false,
  showspaces=false,
  %morestring=[b]',
  %morestring=[b]"
}
\lstnewenvironment{code}[1][]%
  {\minipage{\linewidth} 
   \lstset{basicstyle=\ttfamily\footnotesize,frame=single,#1}}
  {\endminipage}


% Worksheet commands
\newcommand{\worksheetstart}[5]{ %Title, Revision, Date, Author, rootpath
	\ifthenelse{\equal{\worksheetcollection}{false}}{
		\newcommand{\rootpath}{#5}
		\documentheader
		\chapter{#1}
	}{
		\chapter{#1}
	}
	\vspace{-1em}
	\textbf{\tiny Revision #2 at #3. Written by #4}\\
	\vspace{2em}\\
}

\newcommand{\worksheetend}{
	\ifthenelse{\equal{\worksheetcollection}{false}}{
		\collectionend
	}{}
}

\newcommand{\documentheader}{
	\input{\rootpath/setup/tikz-commands.tex}
	\begin{document}
	%\renewcommand{\chaptername}{Worksheet}
	\chapterstyle{section}
	\renewcommand{\beforechapskip}{0pt}
	\renewcommand{\afterchapskip}{0pt}
}

\newcommand{\collectionstart}[1]{
	\newcommand{\rootpath}{#1}
	\renewcommand{\worksheetcollection}{true}
	\documentheader
	\frontmatter
	%\forside
	\input{\rootpath/worksheets/titlepage/titlepage}
	\input{\rootpath/worksheets/preface/preface}
	%\input{\rootpath/worksheets/forord/forord}
	\newpage
	\newpage
	\tableofcontents*
	\mainmatter
}

\newcommand{\collectionend}{
	\backmatter
	\chapter{List of Acronyms}\vspace{3em}
	\input{\rootpath/setup/acronyms}
	\bibliography{\rootpath/setup/bibliography}
	\end{document}
}


\newcommand{\bscode}{
	\lstinline
}

\newcommand{\bscodemath}[1]{
	\text{\lstinline|#1|}
}

\newcommand{\bsqoute}[2]{
	\begin{quote}
		\textit{``#1''}
		\begin{center}
			-- \emph{#2}
		\end{center}
	\end{quote}
}


\newcommand{\lag}{\langle}
\newcommand{\rag}{\rangle}
\newcommand{\besk}[1]{\ensuremath{\lag #1 \rag}}

\newcommand{\namedtodo}[5]
{
  \ifthenelse{\equal{#1}{}}
  {
    \todo[color=#4,caption=
    {\textbf{#3: } #2}]
    {\color{#5}\textbf{#3: }#2}
  }
  {
    \todo[color=#4,caption=
    {\textbf{#3: } #1}
    ,inline]
    {\color{#5}\textbf{#3: }#2}
  }
}
\newcommand{\andreas}[2][]{\namedtodo{#1}{#2}{Andreas}{blue!50!red!10}{black}}
\newcommand{\lone}[2][]{\namedtodo{#1}{#2}{Lone}{orange}{black}}
\definecolor{babypink}{rgb}{0.96, 0.76, 0.76}
\newcommand{\toby}[2][]{\namedtodo{#1}{#2}{Tobias}{babypink}{black}}
\newcommand{\kasper}[2][]{\namedtodo{#1}{#2}{Kasper}{green}{black}}

%multicol
\usepackage{multicol}

% todonotes
%\usepackage[disable]{todonotes} %For final report
\usepackage{todonotes} %For writing notes
\usepackage{fancyvrb}

%Loading AAU macro
\usepackage{lastpage}
%%%%%%%%%%%%%%%%%%%%%%%%%%%%%%%%%%%%%%%%%%%%%%%%
% Macros for the titlepage
%%%%%%%%%%%%%%%%%%%%%%%%%%%%%%%%%%%%%%%%%%%%%%%%
%Creates the aau titlepage
\newcommand{\aautitlepage}[3]{%
  {
    %set up various length
    \ifx\titlepageleftcolumnwidth\undefined
      \newlength{\titlepageleftcolumnwidth}
      \newlength{\titlepagerightcolumnwidth}
    \fi
    \setlength{\titlepageleftcolumnwidth}{0.5\textwidth-\tabcolsep}
    \setlength{\titlepagerightcolumnwidth}{\textwidth-2\tabcolsep-\titlepageleftcolumnwidth}
    %create title page
    \thispagestyle{empty}
    \noindent%
    \begin{tabular}{@{}ll@{}}
      \parbox{\titlepageleftcolumnwidth}{
        \iflanguage{danish}{%
          \includegraphics[width=\titlepageleftcolumnwidth]{titlepage/figures/aau_logo_da}
        }{%
          \includegraphics[width=\titlepageleftcolumnwidth]{titlepage/figures/aau_logo_en}
        }
      } &
      \parbox{\titlepagerightcolumnwidth}{\raggedleft\small
        #2
      }\bigskip\\
       #1 &
      \parbox[t]{\titlepagerightcolumnwidth}{%
      \textbf{Abstract:}\bigskip\par
        \fbox{\parbox{\titlepagerightcolumnwidth-2\fboxsep-2\fboxrule}{%
          #3
        }}
      }\\
    \end{tabular}
    \vfill  
    \clearpage
  }
}

% Environment for problem statements
% Can be auto referenced.
\newtheorem{problem}{Problem}
\def\problemautorefname{Problem}

%Create english project info
\newcommand{\englishprojectinfo}[6]{%
  \parbox[t]{\titlepageleftcolumnwidth}{
    \textbf{Title:}\\ #1\bigskip\par
    %\textbf{Theme:}\\ #2\bigskip\par
    \textbf{Project Period:}\\ #2\bigskip\par
    \textbf{Project Group:}\\ #3\bigskip\par
    \textbf{Participants:}\\ #4\bigskip\par
    \textbf{Supervisor:}\\ #5\bigskip\par
    %\textbf{Copies:} #6\bigskip\par
    \textbf{Page Numbers:} \pageref{LastPage}\bigskip\par
    \textbf{Date of Completion:}\\ #6
  }
}



%Create danish project info
%\newcommand{\danishprojectinfo}[7]{%
 % \parbox[t]{\titlepageleftcolumnwidth}{
 %   \textbf{Titel:}\\ #1\bigskip\par
%    %\textbf{Tema:}\\ #2\bigskip\par
%    \textbf{Projektperiode:}\\ #2\bigskip\par
%    \textbf{Projektgruppe:}\\ #4\bigskip\par
%    \textbf{Deltager(e):}\\ #5\bigskip\par
 %   \textbf{Vejleder(e):}\\ #6\bigskip\par
%    \textbf{Oplagstal:} #7\bigskip\par
   % \textbf{Sidetal:} \pageref{LastPage}\bigskip\par
  %  \textbf{Afleveringsdato:}\\ #8
 % }
%}
}\makeatother
%\worksheetstart{Titlepage}{0}{December 31, 2012}{../../}
\begin{titlingpage}
\aautitlepage{%
  \englishprojectinfo{
    Investigation into hyped concurrency models %title
  }{%
    Fall Semester 2014 %project period
  }{%
    dpt907e14  % project group
  }{%
    %list of group members
    Tobias Ugleholdt Hansen\\
    Andreas Pørtner Karlsen\\ 
    Kasper Breinholt Laurberg\\
  }{%
    %list of supervisors
     Lone Leth Thomsen
  }{%
    \today % date of completion
  }%
}{%department and address
  \textbf{Department of Computer Science}\\
  Selma Lagerløfs Vej 300\\
  DK-9220 Aalborg Ø\\
  \href{http://www.cs.aau.dk}{http://www.cs.aau.dk}
}{% the abstract
Some nice abstract
}
\end{titlingpage}
	\makeatletter \@ifundefined{rootpath}{% Manual to memoir http://mirrors.dotsrc.org/ctan/macros/latex/contrib/memoir/memman.pdf

%\documentclass[a4paper,12pt,fleqn,openany,twoside]{memoir} %two sides for printing
\documentclass[a4paper,12pt,fleqn,openany,oneside]{memoir} %one side for pdf
\usepackage[english]{babel}
\usepackage[utf8]{inputenc}
\usepackage{microtype}
\usepackage{paralist}

%Definitions
\usepackage{amsthm}
\theoremstyle{plain}
\newtheorem{thm}{Theorem}[chapter] % reset theorem numbering for each chapter
\theoremstyle{definition}
\newtheorem{defn}[thm]{Definition}

% Choses the depth of numerations
\setsecnumdepth{subsection}

% Choses the depth of toc
%\maxtocdepth{subsection}

% LaTeX logical statements
\usepackage{ifthen}

% Fancy space after use of e.g. command
\usepackage{xspace}

% Skips after paragraphs
\usepackage{parskip}

% Layout settings
\setlength{\parindent}{0cm}
\setlength{\parskip}{2ex plus 2ex} %kan udvides til f.eks: '2ex plus 2ex minus 0ex'

\sloppybottom

% Don't make a collection per default
\newcommand{\worksheetcollection}{false}

% Bibtex
\usepackage[square,numbers,sort,comma]{natbib}
%\usepackage{cite}
%\bibliographystyle{plainnat}
\bibliographystyle{IEEEtran}


% Fixmes
\usepackage{fixme}
\fxsetup{draft}

% Mathematic
\usepackage{amsmath}
\usepackage{amsfonts}
\usepackage{amssymb}
\usepackage{stmaryrd}
\allowdisplaybreaks[1]


% Acronyms
\usepackage[printonlyused]{acronym}

% Images
\usepackage{graphicx}
\usepackage{wrapfig}
\usepackage[outdir=./]{epstopdf}
\usepackage{epsfig}


% Captions ans subcaptions
\captionnamefont{\footnotesize\bfseries}
\captiontitlefont{\footnotesize}

% Enable memoir subfloats for figures and tables
\newsubfloat{figure}
\newsubfloat{table}

% Hack memoir subfigure styles to have bold label and footnotesize fonts
\renewcommand{\thesubfigure}{\footnotesize\bfseries{(\alph{subfigure})}}
\renewcommand{\thesubtable}{\footnotesize\bfseries{(\alph{subtable})}}

\renewcommand{\subcaption}[2][]{\subbottom[\footnotesize{#1}]{#2}}

% Memoir tweak pagenumbers
%\pagestyle{headings}

% Tikz
\usepackage{tikz}
\usetikzlibrary{arrows,shapes,calc,positioning}
\pgfmathsetseed{1}

%Pgf plots
\usepackage{pgfplots}
\pgfplotsset{compat=1.5}
% loatbarrier, keep figures within (sub,subsub) sections
\usepackage{placeins}
\usepackage{pgfplots}
\usepgfplotslibrary{units}
\usepackage[space-before-unit,range-units = repeat]{siunitx}

% Hyperlinked auto references
\usepackage[hidelinks]{hyperref}
\usepackage[nameinlink]{cleveref}
\crefname{lstlisting}{Listing}{Listings}  
\Crefname{lstlisting}{Listing}{Listings}
%\def\chapterautorefname{Kapitel}
%\def\sectionautorefname{Afsnit}
%\def\subsectionautorefname{Afsnit}
%\def\subsubsectionautorefname{Underafsnit}
%\def\figureautorefname{Figur}
%\def\lstlistingautorefname{Listing}
%\def\lstnumberautorefname{Linje}
%\def\itemautorefname{Punkt}
\usepackage[hypcap]{caption} % Link to top of the figure and not the caption

%Sick shit to make \Autoref command
%http://tex.stackexchange.com/questions/36575/autorefs-inserted-text-has-not-the-correct-case
\def\HyLang@english{%
  \def\equationautorefname{Equation}%
  \def\footnoteautorefname{Footnote}%
  \def\itemautorefname{item}%
  \def\figureautorefname{Figure}%
  \def\tableautorefname{Table}%
  \def\partautorefname{Part}%
  \def\appendixautorefname{Appendix}%
  \def\chapterautorefname{Chapter}%
  \def\sectionautorefname{Section}%
  \def\subsectionautorefname{Subsection}%
  \def\subsubsectionautorefname{Subsubsection}%
  \def\paragraphautorefname{Paragraph}%
  \def\subparagraphautorefname{Subparagraph}%
  \def\FancyVerbLineautorefname{Line}%
  \def\theoremautorefname{Theorem}%
  \def\pageautorefname{Page}%
}

% Reference greencommentssections with number and name
\usepackage{nameref}
\newcommand{\bsnameref}[1]{\Cref{#1} ``\nameref{#1}''}
\newcommand{\bsref}[1]{\Cref{#1}}
\newcommand{\bsbilagref}[1]{Appendix \ref{#1}}
\newcommand{\bsbilagnameref}[1]{Appendix \ref{#1} ``\nameref{#1}''}
\newcommand{\pling}[1]{``#1''}


% Listings for code qoutes
\usepackage{listings}
%\usepackage[usenames,dvipsnames,svgnames,table]{xcolor}
\usepackage{color}
\usepackage{xcolor}
\definecolor{bluekeywords}{rgb}{0.13,0.13,1}
\definecolor{greencomments}{rgb}{0,0.5,0}
\definecolor{redstrings}{rgb}{0.9,0,0}
\usepackage{caption} 
\usepackage{multicol}
\DeclareCaptionFont{white}{\color{white}}
\DeclareCaptionFormat{listing}{\colorbox{gray}{\parbox{\textwidth}{#1#2#3}}}
\captionsetup[lstlisting]{format=listing,labelfont=white,textfont=white}
%\lstset{numbers=left}
\lstset{
	basicstyle=\footnotesize,
	tabsize=2,
	breaklines=true,
  literate={æ}{{\ae}}1 {ø}{{\o}}1 {å}{{\aa}}1 {Æ}{{\AE}}1 {Ø}{{\O}}1 {Å}{{\AA}}1,
  keywords={typeof, new, true, false, catch, function, return, null, catch, switch, var, if, in, while, do, else, case, break},
  keywordstyle=\color{blue}\bfseries,
  ndkeywords={class, export, boolean, throw, implements, import, this},
  ndkeywordstyle=\color{darkgray}\bfseries,
  identifierstyle=\color{black},
  sensitive=false,
  comment=[l]{//},
  morecomment=[s]{/*}{*/},
  commentstyle=\color{purple}\ttfamily,
  stringstyle=\color{red}\ttfamily,
  numbers=left,
  numbersep=-5pt,
  showstringspaces=false,
  showspaces=false,
  %morestring=[b]',
  %morestring=[b]"
}
\lstnewenvironment{code}[1][]%
  {\minipage{\linewidth} 
   \lstset{basicstyle=\ttfamily\footnotesize,frame=single,#1}}
  {\endminipage}


% Worksheet commands
\newcommand{\worksheetstart}[5]{ %Title, Revision, Date, Author, rootpath
	\ifthenelse{\equal{\worksheetcollection}{false}}{
		\newcommand{\rootpath}{#5}
		\documentheader
		\chapter{#1}
	}{
		\chapter{#1}
	}
	\vspace{-1em}
	\textbf{\tiny Revision #2 at #3. Written by #4}\\
	\vspace{2em}\\
}

\newcommand{\worksheetend}{
	\ifthenelse{\equal{\worksheetcollection}{false}}{
		\collectionend
	}{}
}

\newcommand{\documentheader}{
	\input{\rootpath/setup/tikz-commands.tex}
	\begin{document}
	%\renewcommand{\chaptername}{Worksheet}
	\chapterstyle{section}
	\renewcommand{\beforechapskip}{0pt}
	\renewcommand{\afterchapskip}{0pt}
}

\newcommand{\collectionstart}[1]{
	\newcommand{\rootpath}{#1}
	\renewcommand{\worksheetcollection}{true}
	\documentheader
	\frontmatter
	%\forside
	\input{\rootpath/worksheets/titlepage/titlepage}
	\input{\rootpath/worksheets/preface/preface}
	%\input{\rootpath/worksheets/forord/forord}
	\newpage
	\newpage
	\tableofcontents*
	\mainmatter
}

\newcommand{\collectionend}{
	\backmatter
	\chapter{List of Acronyms}\vspace{3em}
	\input{\rootpath/setup/acronyms}
	\bibliography{\rootpath/setup/bibliography}
	\end{document}
}


\newcommand{\bscode}{
	\lstinline
}

\newcommand{\bscodemath}[1]{
	\text{\lstinline|#1|}
}

\newcommand{\bsqoute}[2]{
	\begin{quote}
		\textit{``#1''}
		\begin{center}
			-- \emph{#2}
		\end{center}
	\end{quote}
}


\newcommand{\lag}{\langle}
\newcommand{\rag}{\rangle}
\newcommand{\besk}[1]{\ensuremath{\lag #1 \rag}}

\newcommand{\namedtodo}[5]
{
  \ifthenelse{\equal{#1}{}}
  {
    \todo[color=#4,caption=
    {\textbf{#3: } #2}]
    {\color{#5}\textbf{#3: }#2}
  }
  {
    \todo[color=#4,caption=
    {\textbf{#3: } #1}
    ,inline]
    {\color{#5}\textbf{#3: }#2}
  }
}
\newcommand{\andreas}[2][]{\namedtodo{#1}{#2}{Andreas}{blue!50!red!10}{black}}
\newcommand{\lone}[2][]{\namedtodo{#1}{#2}{Lone}{orange}{black}}
\definecolor{babypink}{rgb}{0.96, 0.76, 0.76}
\newcommand{\toby}[2][]{\namedtodo{#1}{#2}{Tobias}{babypink}{black}}
\newcommand{\kasper}[2][]{\namedtodo{#1}{#2}{Kasper}{green}{black}}

%multicol
\usepackage{multicol}

% todonotes
%\usepackage[disable]{todonotes} %For final report
\usepackage{todonotes} %For writing notes
\usepackage{fancyvrb}

%Loading AAU macro
\usepackage{lastpage}
%%%%%%%%%%%%%%%%%%%%%%%%%%%%%%%%%%%%%%%%%%%%%%%%
% Macros for the titlepage
%%%%%%%%%%%%%%%%%%%%%%%%%%%%%%%%%%%%%%%%%%%%%%%%
%Creates the aau titlepage
\newcommand{\aautitlepage}[3]{%
  {
    %set up various length
    \ifx\titlepageleftcolumnwidth\undefined
      \newlength{\titlepageleftcolumnwidth}
      \newlength{\titlepagerightcolumnwidth}
    \fi
    \setlength{\titlepageleftcolumnwidth}{0.5\textwidth-\tabcolsep}
    \setlength{\titlepagerightcolumnwidth}{\textwidth-2\tabcolsep-\titlepageleftcolumnwidth}
    %create title page
    \thispagestyle{empty}
    \noindent%
    \begin{tabular}{@{}ll@{}}
      \parbox{\titlepageleftcolumnwidth}{
        \iflanguage{danish}{%
          \includegraphics[width=\titlepageleftcolumnwidth]{titlepage/figures/aau_logo_da}
        }{%
          \includegraphics[width=\titlepageleftcolumnwidth]{titlepage/figures/aau_logo_en}
        }
      } &
      \parbox{\titlepagerightcolumnwidth}{\raggedleft\small
        #2
      }\bigskip\\
       #1 &
      \parbox[t]{\titlepagerightcolumnwidth}{%
      \textbf{Abstract:}\bigskip\par
        \fbox{\parbox{\titlepagerightcolumnwidth-2\fboxsep-2\fboxrule}{%
          #3
        }}
      }\\
    \end{tabular}
    \vfill  
    \clearpage
  }
}

% Environment for problem statements
% Can be auto referenced.
\newtheorem{problem}{Problem}
\def\problemautorefname{Problem}

%Create english project info
\newcommand{\englishprojectinfo}[6]{%
  \parbox[t]{\titlepageleftcolumnwidth}{
    \textbf{Title:}\\ #1\bigskip\par
    %\textbf{Theme:}\\ #2\bigskip\par
    \textbf{Project Period:}\\ #2\bigskip\par
    \textbf{Project Group:}\\ #3\bigskip\par
    \textbf{Participants:}\\ #4\bigskip\par
    \textbf{Supervisor:}\\ #5\bigskip\par
    %\textbf{Copies:} #6\bigskip\par
    \textbf{Page Numbers:} \pageref{LastPage}\bigskip\par
    \textbf{Date of Completion:}\\ #6
  }
}



%Create danish project info
%\newcommand{\danishprojectinfo}[7]{%
 % \parbox[t]{\titlepageleftcolumnwidth}{
 %   \textbf{Titel:}\\ #1\bigskip\par
%    %\textbf{Tema:}\\ #2\bigskip\par
%    \textbf{Projektperiode:}\\ #2\bigskip\par
%    \textbf{Projektgruppe:}\\ #4\bigskip\par
%    \textbf{Deltager(e):}\\ #5\bigskip\par
 %   \textbf{Vejleder(e):}\\ #6\bigskip\par
%    \textbf{Oplagstal:} #7\bigskip\par
   % \textbf{Sidetal:} \pageref{LastPage}\bigskip\par
  %  \textbf{Afleveringsdato:}\\ #8
 % }
%}
}\makeatother
\worksheetstart{Preface}{1}{December 17, 2013}{Mino}{../../}
This report documents project work done by group dpt907e14 at the Department of Computer Science at Aalborg University. The report was written as part of the Computer Science (IT) study program in the fall of 2014 at the 9th semester.

The first time an acronym is used it will appear in the format: Threads \& Locks (TL). Inline quotations and names will appear in \textit{italics}. The work presented in this report is based on work or results described in books, articles, video lectures and research papers from outside sources. The full list of acronyms along with the bibliography and appendix, can be found at the end of the report.

One chapter contains a number of mathematical formulas. The notation and intuition of these formulas will be defined after the formula. The exact understanding of the formulas is not important, it is the understanding of the amount of work the formulas will produce when calculated on a computer.

The report was written under the assumption, that the reader has solid understanding of computer science, and basic knowledge of concurrency. The basics of concurrency will be explained in \bsref{chp:con_basics}, and can be skipped if the reader already know this. The report can be read from end to end as a coherent piece, or as the reading guide below suggests.

Chapter 4, 5, and 6 can be read individually to obtain knowledge about a specific concurrency model. Chapter 7 compares the characteristics of the concurrency models and can be read to gain an overview of its characteristic differences. Chapter 8 can be read individually to see the performance result between the models. Chapter 9 can be used as a guide for choosing concurrency model, and contains only a summary of the most important factors.

We would like to give a special thanks to our supervisor Lone Leth Thomsen, from the Department of Computer Science, for excellent guidance, immaculate attention to detail. She supplied invaluable help throughout the project with professionalism and black humour. Her constructive criticism helped us to narrow down the subject and kept us motivated and enthusiastic about the project.


\newpage
\vspace*{30 mm}
%\vspace*{\fill}
\begin{vplace}

\begin{minipage}[b]{0.45\textwidth}
 \centering
 \rule{\textwidth}{0.5pt}\\
  Tobias Ugleholdt Hansen\\
 {\footnotesize tuha13@student.aau.dk}
\end{minipage}
\begin{minipage}[b]{0.45\textwidth}
 \centering
 \rule{\textwidth}{0.5pt}\\
  Andreas Pørtner Karlsen\\
 {\footnotesize akarls13@student.aau.dk}
\end{minipage}\\\\
\begin{minipage}[b]{0.45\textwidth}
 \centering
 \rule{\textwidth}{0.5pt}\\
  Kasper Breinholt Laurberg\\
 {\footnotesize klaurb13@student.aau.dk}
\end{minipage}\\\\


\end{vplace}
\worksheetend

	%\input{\rootpath/worksheets/forord/forord}
	\newpage
	\newpage
	\tableofcontents*
	\mainmatter
}

\newcommand{\collectionend}{
	\backmatter
	\chapter{List of Acronyms}\vspace{3em}
	\begin{acronym}[ITU-T]
\acro{WS}{Webservice}
\end{acronym}

	\bibliography{\rootpath/setup/bibliography}
	\end{document}
}


\newcommand{\bscode}{
	\lstinline
}

\newcommand{\bscodemath}[1]{
	\text{\lstinline|#1|}
}

\newcommand{\bsqoute}[2]{
	\begin{quote}
		\textit{``#1''}
		\begin{center}
			-- \emph{#2}
		\end{center}
	\end{quote}
}


\newcommand{\lag}{\langle}
\newcommand{\rag}{\rangle}
\newcommand{\besk}[1]{\ensuremath{\lag #1 \rag}}

\newcommand{\namedtodo}[5]
{
  \ifthenelse{\equal{#1}{}}
  {
    \todo[color=#4,caption=
    {\textbf{#3: } #2}]
    {\color{#5}\textbf{#3: }#2}
  }
  {
    \todo[color=#4,caption=
    {\textbf{#3: } #1}
    ,inline]
    {\color{#5}\textbf{#3: }#2}
  }
}
\newcommand{\andreas}[2][]{\namedtodo{#1}{#2}{Andreas}{blue!50!red!10}{black}}
\newcommand{\lone}[2][]{\namedtodo{#1}{#2}{Lone}{orange}{black}}
\definecolor{babypink}{rgb}{0.96, 0.76, 0.76}
\newcommand{\toby}[2][]{\namedtodo{#1}{#2}{Tobias}{babypink}{black}}
\newcommand{\kasper}[2][]{\namedtodo{#1}{#2}{Kasper}{green}{black}}

%multicol
\usepackage{multicol}

% todonotes
%\usepackage[disable]{todonotes} %For final report
\usepackage{todonotes} %For writing notes
\usepackage{fancyvrb}

%Loading AAU macro
\usepackage{lastpage}
%%%%%%%%%%%%%%%%%%%%%%%%%%%%%%%%%%%%%%%%%%%%%%%%
% Macros for the titlepage
%%%%%%%%%%%%%%%%%%%%%%%%%%%%%%%%%%%%%%%%%%%%%%%%
%Creates the aau titlepage
\newcommand{\aautitlepage}[3]{%
  {
    %set up various length
    \ifx\titlepageleftcolumnwidth\undefined
      \newlength{\titlepageleftcolumnwidth}
      \newlength{\titlepagerightcolumnwidth}
    \fi
    \setlength{\titlepageleftcolumnwidth}{0.5\textwidth-\tabcolsep}
    \setlength{\titlepagerightcolumnwidth}{\textwidth-2\tabcolsep-\titlepageleftcolumnwidth}
    %create title page
    \thispagestyle{empty}
    \noindent%
    \begin{tabular}{@{}ll@{}}
      \parbox{\titlepageleftcolumnwidth}{
        \iflanguage{danish}{%
          \includegraphics[width=\titlepageleftcolumnwidth]{titlepage/figures/aau_logo_da}
        }{%
          \includegraphics[width=\titlepageleftcolumnwidth]{titlepage/figures/aau_logo_en}
        }
      } &
      \parbox{\titlepagerightcolumnwidth}{\raggedleft\small
        #2
      }\bigskip\\
       #1 &
      \parbox[t]{\titlepagerightcolumnwidth}{%
      \textbf{Abstract:}\bigskip\par
        \fbox{\parbox{\titlepagerightcolumnwidth-2\fboxsep-2\fboxrule}{%
          #3
        }}
      }\\
    \end{tabular}
    \vfill  
    \clearpage
  }
}

% Environment for problem statements
% Can be auto referenced.
\newtheorem{problem}{Problem}
\def\problemautorefname{Problem}

%Create english project info
\newcommand{\englishprojectinfo}[6]{%
  \parbox[t]{\titlepageleftcolumnwidth}{
    \textbf{Title:}\\ #1\bigskip\par
    %\textbf{Theme:}\\ #2\bigskip\par
    \textbf{Project Period:}\\ #2\bigskip\par
    \textbf{Project Group:}\\ #3\bigskip\par
    \textbf{Participants:}\\ #4\bigskip\par
    \textbf{Supervisor:}\\ #5\bigskip\par
    %\textbf{Copies:} #6\bigskip\par
    \textbf{Page Numbers:} \pageref{LastPage}\bigskip\par
    \textbf{Date of Completion:}\\ #6
  }
}



%Create danish project info
%\newcommand{\danishprojectinfo}[7]{%
 % \parbox[t]{\titlepageleftcolumnwidth}{
 %   \textbf{Titel:}\\ #1\bigskip\par
%    %\textbf{Tema:}\\ #2\bigskip\par
%    \textbf{Projektperiode:}\\ #2\bigskip\par
%    \textbf{Projektgruppe:}\\ #4\bigskip\par
%    \textbf{Deltager(e):}\\ #5\bigskip\par
 %   \textbf{Vejleder(e):}\\ #6\bigskip\par
%    \textbf{Oplagstal:} #7\bigskip\par
   % \textbf{Sidetal:} \pageref{LastPage}\bigskip\par
  %  \textbf{Afleveringsdato:}\\ #8
 % }
%}
}\makeatother
\worksheetstart{Introduction}{1}{September 5, 2014}{Kasper}{../../}
In \bsref{sec:motivation} we describe the motivation of the work behind this report. We account for the related work in \bsref{sec:relatedwork}. We conduct a preliminary investigation of concurrency models in \bsref{sec:prelim}. Finally, we describe the problem statement in \bsref{sec:problemstatement}.
\label{chap:intro}

\section{Motivation}
% Moores law, skalér med flere kerner, ikke højere CPU
% Krav til at applikationer kræver mere ydelse
% Focus på skalering til flere computere i.e. parallel computing, ikke kun flere kerner
% Brug for en evolution, inden for tilgange til at skrive parallele programmer
% Vi vil kigge på concurrency inden for rammerne af samme maskine
Moore's law\cite{moore1965cramming} is the empirical observation that the number of transistors per area, on a integrated circuit, was approximately doubled roughly every 18 months\cite[p. 203]{mack2011fifty}. As a result, processing speed kept increasing as more transistors where added. This increase has however stagnated and processing power has mainly increased in the form of additional processing cores, as opposed to the speed of each of these processing cores\cite[p. 22]{sevenModels}. That is, computers have moved from having a \ac{CPU} consisting of one single core to one having multiple cores. This tendency is displayed in \bsref{fig:moores_in_reality}. 
\label{sec:motivation}

\begin{figure}[htbp]
\centering
 \includegraphics[width=0.9\textwidth]{\rootpath/worksheets/introduction/figures/moores_core_vs_frequency} 
 \caption{A comparison of the development of Transistors, Frequency, and Cores\cite{isca2009}. A change of the tendency occurred around 2005. The overall number of transistors has kept increasing, but the increase in MHz, has been substituted by an increase in the number of cores.}
\label{fig:moores_in_reality}
\end{figure}

Programming languages developed while it was believed that single core performance would keep increasing, such as assembly and C, were closely related to the Von Neumann architecture. As a consequence, the programming model was designed for synchronous execution. This introduces a challenge in how programs should be written to execute efficiently on a multi-core processor. A lot of effort has been invested into identifying suitable ways of programming against such architectures: With inspiration from the database domain, transactions have been integrated into software, known as \ac{STM}\cite{shavit1997software}\cite{scherer2005advanced}. \ac{FP} is having a renaissance, and is now being used outside the world of academia, for instance in the financial sector\cite{minsky2008caml}. Another proposal is the Actor model\cite{hewitt1973universal}, that with the rise of multi-core processors has gained a lot of popularity in different languages  \cite{haller2007actors}\cite{hewitt2014actor}. 

The purpose of this report is to compare different ways of concurrent programming with a focus on their implementation efficiency and characteristics. The focus will be on concurrency models that have seen widespread use, as well as models that currently are receiving academic and industrial interest. The result will assist programmers to choose which concurrency model to use given the wanted properties of their project.

\section{Related Work}\label{sec:relatedwork}
% Formål med preliminary analysis
This section contains an analysis of work related to concurrency. It will be conducted by investigating papers, articles, and other research material of relevance, in order to discover state of the art, and open problems within the area of concurrency models. The purpose of the analysis is to establish an overview which will be used to choose the further path of our investigation.
% Moores law, skalér med flere kerner, ikke højere CPU
% Krav til at applikationer kræver mere ydelse
% Focus på skalering til flere computere i.e. parallel computing, ikke kun flere kerner
% Brug for en evolution, inden for tilgange til at skrive parallele programmer
% Vi vil kigge på concurrency inden for rammerne af samme maskine

%Threads and locks duer ikke
%Hvad er der behov for
%Hvad andet findes der, der opfylder behovet
In ``The Free Lunch is Over''\cite{sutter2005free}, it is claimed that the era of gaining performance increase without changing a program is over:

\bsqoute{[...] if you want your application to benefit from the continued exponential throughput advances in new processors, it will need to be a well-written concurrent (usually multithreaded) application. And that’s easier said than done, because not all problems are inherently parallelizable and because concurrent programming is hard.}{Herb Sutter}

In other words, hardware limitations in the processor development are having an impact on the way software should be developed to utilize the full potential of multi-core processors. The key to take advantage of the raw performance of future multicore processors, is to enable the programmer to harvest that performance\cite[p. 31]{asanovic2006landscape}. In ``The landscape of Parallel Computing Research''\cite{asanovic2006landscape} they propose that a programming model needs to enable productivity and implementation efficiency\cite[p. 31]{asanovic2006landscape}. Productivity is defined as the speed a programmer can develop a program, and implementation efficiency is defined as the runtime speed.

In resent years research has gone into designing languages and language constructs which allows the programmer to easily scale an application over multiple cores or even machine. One of the languages designed is X10\cite{charles2005x10}. X10 encompasses a language construct that allows the programmer to talk about where a statement should be executed, using the notion of places\cite[p. 54]{tardieu2014x10}. Combined with specification of either synchronous or asynchronous this construct allows the programmer to spawn tasks to execute on other machine. Spawning asynchronous tasks on the same machine is also possible\cite[p. 55]{tardieu2014x10}. X10 has been shown to perform between 41\% and 87\% efficiency of the highly optimized IBM HPC class 1 benchmark C implementation\cite[p. 62]{tardieu2014x10}.

Existing language has also gained construct for scaling applications over multiple cores and machines.  As an example, Cloud Haskell extends Haskell which construct for asynchronous message parsing as inspired by Erlang\cite[p. 119]{epstein2011towards}. It further extends this idea with communication over statically typed channel\cite[p. 122]{epstein2011towards}. Cloud Hashell has been shown to scale better than the Hadoop MapReduce framework\cite[p. 128]{epstein2011towards}.

Productivity and implementation efficiency of a concurrency model should both be as high as possible, since they are both positive effects. While the wish to enable both productivity and implementation efficiency seems reasonable, the two goals are often conflicting. An abstraction over concurrency mechanisms frees the programmer from low level details, and increases productivity. However, by using the abstraction, the programmer loses the ability to fine tune the underlying concurrency implementation for performance. A concurrency model must therefore seek to find the right abstraction, that gives the programmer enough abstraction for a productivity increase, without losing too much implementation efficiency.

With these discoveries taken into account, a concurrency model with abstraction over low level details, and solid runtime efficiency is of special interest for our investigation.

\section{Preliminary Investigation}\label{sec:prelim}
The purpose of this section is to document our preliminary concurrency model analysis, in order to establish a foundation on which we later on will be able to choose the models to analyse further. For each model, a reason for looking into it will be given followed by a short description, which will provide an overview of the model. Later the chosen models will be presented further in depth.

\andreas[inline]{Lav en reference til den uddybende forklaring for de enkelte modeller når den er færdig}
\subsection{\acl{TL}}
\label{subsec:tl}
The \ac{TL} model has been chosen based on its historical significance and widespread commercial use\cite[p. 58]{sutter2005software}. The traditional \ac{TL} approach is based on shared memory and uses locks to limit the access to critical regions in order to ensure correct execution\cite[p. 1]{saha2006mcrt}. The use of threads and locks leads to a number of issues, including: deadlocks, difficult fined-grained synchronization and no support for error recovery\cite[p. 187]{saha2006mcrt}. The \ac{TL} model has been implemented in many languages, including C and Java. 

A problem of particular importance in modern software development is composition of code segments. Many programs rely on libraries to handle part of its operations. Using the \ac{TL} approach it is however not guarantied that combining two lock based code segments will result in a correct program\cite[p. 56]{sutter2005software}.

Using the \ac{TL} approach it is left up to the programmer to identity correct lock placements as well as balancing lock granularity vs performance\cite[p. 49]{harris2005composable}. \ac{TL} concurrency is generally believed to be hard to get right\cite[p. 92]{herlihy2003software}. Arguments have been made for the case that the \ac{TL} approach is insufficient for today's concurrency needs and that new models which put less strain on the programmer are needed\cite[p. 3]{jones2007beautiful}\cite[p. 48]{harris2005composable}.


\subsection{Actor model}
%Formalism-centric programming models, such as Actors [Hewitt et al 1973], try to reduce the chance of programmer making mistakes by having clean semantics and offer the chance to remove bugs by verifying correctness of portions of the code. "A view from berkeley, p. 33" - Toby: Maybe use this in the conclusion of which model to choose
The actor model was developed by Carl Hewitt and his team at MIT in 1973\cite{hewitt1973universal}. The goal was to simplify building concurrent systems and especially ease developers in reasoning about concurrent programs, to be more certain that programs worked as intended\cite[p. 14]{haller2012actors}. Since the goal of the Actor model seeks to solve well known problems of concurrency, which we describe further in \bsref{sec:tl_ci}, the model is a candidate of interest for further investigation. The actor model is a general model for concurrency that can be used with almost any language and it is often known as the concurrency model of Erlang\footnote{\url{http://erlang.org}} where it is directly implemented in the language\cite[p. 115]{sevenModels}. Other languages that implement the actor model, either directly or indirectly, include Scala, Smalltalk, Java, C++\footnote{http://c2.com/cgi/wiki?ActorLanguages}. In addition to the goal of the Actor model,  the popularity and widespread use of the Actor model also contributes to our interest to the model.

The idea of the actor model is to use actors as the fundamental unit of computation. An actor has the following essential elements of computation\cite{actorLangNextVideo}: Processing, storage, and communication. The actor encapsulates state and communicates with other actors through asynchronous message passing. In response to a message an actor can\cite{hewitt2014actor}:
\begin{itemize}
\item Create new actors
\item Send messages to actors it knows
\item Modify internal behavior (how the next message it receives should be handled)
\end{itemize}
An actor has a mailbox which messages sent to the actor arrive in. The actor dequeues the mailbox and processes one message at a time\footnote{Not true for recursion\cite{actorLangNextVideo}}. Messages sent can take arbitrarily long time to arrive, and if sent concurrently can arrive in a mailbox in any order\cite{hewitt2014actor}.

The actor model is attempting to mitigate the issues related to shared state, for instance deadlocks and race conditions, by avoiding shared state between actors\citep[Chap. 32]{odersky2011programming}. Instead the actor model allows isolated mutable state on actors and rely solely on asynchronous message passing between actors.

\subsection{\acl{Rx}}
At Microsoft Research, Erik Meijer and his team have developed \ac{Rx} which is a library for composing asynchronous and event-based programs. The approach has gained a lot of attention by big players in the industry, which is pushing on the idea to spread to various platforms. For instance Netflix have ported it to Java\cite{RxJava}, and Facebook ported it to JavaScript\cite{react}. Google is also inspired by the idea in their new language Dart\cite{dart}.

\ac{Rx} gives reactive capabilities to mainstream imperative languages, such as Java and C\#\cite{csharp}. The idea is, to abstract over the complexity that asynchronous computations introduces, by providing a way to orchestrate asynchronous data streams in a uniform way, regardless of the underlying concurrency model.

\ac{Rx} provides a way to deal with asynchronous data streams, the same way programmers deal with synchronous data streams. In Java, a synchronous data stream is a data structure which implements the \bscode{Iterable} interface\footnote{\url{http://docs.oracle.com/javase/7/docs/api/java/lang/Iterable.html}}. However, while iterations over an \bscode{Iterable} blocks between each iteration, \ac{Rx} does not block between each push of the next element. This a key difference, since not blocking while doing concurrency computations is essential for performance.

It is based on the idea of the Observer Pattern\cite{gamma1994design}. That is, whenever the subject of observation, called an observable, is changed, the observers are notified, and can react to the change. This is said to be a push based approach, since changes is pushed out to its observers. This is both true for the Observer Pattern and \ac{Rx}. Since \ac{Rx} is intended for asynchronous tasks that can fail, for example network calls, it extends the Observer Pattern and provides a way to notify its observers whenever an error occurred or if the stream ends.

\ac{Rx} can hardly be called a concurrency model itself, since it cannot be used to spawn concurrent tasks. However, it provides a uniform way, to structure asynchronous tasks independently of the underlying concurrency model. The underlying model could be \ac{TL}, actors, or a third alternative. It does not matter, since this is abstracted away from the client, that treats all interaction with Observables as asynchronous.

\subsection{\acl{STM}}
\ac{STM} has been viewed by many people as a promising direction for solving concurrency issues\cite{sutter2005software}. There has been a lot of attention to this area, and there is still active research. Due to this, it is a natural candidate of interest.

\ac{STM} takes an approach to concurrency, that is based on transactions as known from database theory\cite[p. 1]{shavit1997software}. \ac{STM} transitions are atomic and isolated. Atomic meaning that either all writes in a given transaction are committed, or non of them are, while isolated means that from a given transaction \bscode{T}, any other concurrently running transaction will have no impact on the values committed by \bscode{T} when \bscode{T} succeeds\cite[p. 102]{sevenModels}. These properties are similar to the \ac{ACID} properties known from databases\cite[p. 754]{elmasri2011fundamentals}.

\ac{STM} is implemented in a number of languages, including Clojure\cite[p. 101]{sevenModels}, Concurrent Haskell\cite{harris2005composable} and Scala\cite{goodman2011muts} either directly or using a library. Generally the idea is that programmers specify regions which are to be executed as a transaction. The compiler or library then takes care of ensuring that the \ac{STM} transaction principles are maintained and that the transactions are eventually committed\cite[p. 1]{saha2006mcrt}. The transaction will be retried if any other transactions read or modified the used data in the process. It is retried until it is successful.

The strength of \ac{STM} lies in the avoidance of many of the issues mentioned in \bsref{subsec:tl} that plague the traditional \ac{TL} approach. \ac{STM} can avoid deadlocks, priority inversion and eliminates the issue of balancing lock granularity vs performance\cite[p. 1]{harris2005composable}.

\subsection{Communicating Sequential Processes}
The \ac{CSP} model was invented by C. A. R. Hoare\cite{hoare1978communicating} and the paper is widely regarded as one of the most influential papers in computer science\cite{abdallah2005communicating}. 

\ac{CSP} has influenced the design of numerous languages and it is implemented either directly in the language or indirectly in the form of a library. This includes programming languages such as Ada, occam and Concurrent ML\cite{abdallah2005communicating}. The model has still a lot of attention and the recent popularity is especially due to the Go\footnote{https://golang.org/doc/faq\#csp} programming language developed by Google, as Go builds their concurrency model on the ideas of \ac{CSP}\cite[Chap. 6]{sevenModels}. \ac{CSP} is therefore seen as an important and viable candidate for consideration.

A process in \ac{CSP} is a basic construct that operates independently and communication between such processes enables concurrency\cite{ibmCSP}. There is no shared state between processes and communication is accomplished solely through message passing to and from channels. A channel in \ac{CSP} is a first-class queue where messages are added to in one end and removed from in the other\cite[Chap. 6]{sevenModels}. It is possible to have arbitrarily many reader/writer processes on a channel. 

The strength of \ac{CSP} lies in avoiding many of the issues that are related to shared state by modelling computations as independent isolated processes which communicate together through channels. 

% nævn den originalt er en mathematical model?
%Noget om at CSP har ændret sig over tid? Actor har altid været den samme
%Til valg af modeller: CSP og Actor minder meget om hinaden, derfor vælger vi kun at se på Actor (men der er forskelle! evt. vælg at pointere dem ud, eller skriv et lille afsnit om det i Actor afsnittet)
	%Evt. begrund med at vi vælger CSP fra fordi det er mere matematisk orienteret, hvilket ikke er attraktivt for 
	%CSP har fokus på channels (kommunikation mellem entities), hvor Actor har fokus på Actors (entities)
\section{Scope}
Since it is not feasible to investigate every concurrency model existing, we will limit the scope of this report to the concurrency models that with our current knowledge has the most potential to solve known problems of concurrency.

In our preliminary analysis, we briefly covered the traditional \ac{TL} model, as well as the actor model, \ac{Rx}, \ac{STM} and \ac{CSP} approaches. From these models we have chosen to continue our investigation into \ac{TL}, Actors, and \ac{STM}. \ac{TL} has been selected based on its widespread use and historical significance. Actors and \ac{STM} have been chosen based on their up and coming usage or ongoing research efforts. Each of these models also have a vastly different approach to handling concurrency, and will provide varying perspectives. We choose not to investigate \ac{Rx} as it is not a concurrency model, as it cannot spawn concurrent tasks, and therefore does not fit into our investigation. The actor model and CSP are similar models as they both consist of independent concurrently executing entities, processes in CSP and actors in the actor model, that only communicate by sending messages. The important difference between the models is that the actor model has focus on the entities (actors), where CSP has focus on channels used to communicate between entities (processes)\cite[p. 153]{sevenModels}. Due to the similarity of the models we have chosen to look only at one of the models\lone{Forhold jer til det i reflection}. The actor model has been chosen due to its widespread use and popularity\andreas{Kilde}.

In order to analyse the chosen concurrency models, we will do an implementation with each model. To test the implementation efficiency of the models, and not of the language that contains the model, we need to either find a language that contains all of the chosen models, or a common foundation for the languages chosen. 

To our best knowledge, it is not possible to find a single language, that contains native implementations of all the selected concurrency models. This is in spite of extended research of existing languages. Although libraries implementing the models do exist for a variety of languages, they do not yield the same optimization capacities, as languages with a native implementation, which can do compiletime optimizations. Due to this, multiple languages will be employed. 

One language will be chosen for each of the selected models and this language will be used to exemplify the model as well as implementing the performance testcase described in \bsref{chap:performance}. To ensure that our performance test produces comparable results across these languages, a common foundation is needed. Such a foundation is to ensure that the runtime environment of the languages is similar enough to produce comparable results e.g. not comparing the performance of a natively compiled language such as C++ to that of a interpreted language such as Python. We have chosen to limit the languages employed to languages running on the \ac{JVM}. The \ac{JVM} was explicitly designed for the Java language, and a large number of languages, of varying paradigms, target the \ac{JVM} \cite[p. 1]{singer2003jvm}, providing a large basis for selection of languages. The \ac{CLR} was also considered as a common basis. While it is designed for being the target of multiple languages, it does not have the platform independence that the \ac{JVM} offers. An open source implementation, running on Linux, is available trough the mono project\footnote{\url{http://www.mono-project.com/}}. This has however been shown to have significantly worse performance than its Windows counterpart\cite[p. 59]{totoo2012haskell}.\lone{Diskuter valg i refleksion}

\section{Problem Statement}\label{sec:problemstatement}
In order to structure our investigation into concurrency models, we have extracted a number of problem statement questions. The questions and selected models are based on findings presented in our preliminary analysis, described previously. These questions will serve as a guideline for our further investigation, and will be used to conclude upon the project in \bsref{chap:conclusion}.

The questions we seek to answer are:
\begin{enumerate}
\item Which issues exist with the traditional \ac{TL} concurrency model?
\item What are the characteristics of the selected models? Including their strengths and weaknesses.
\item How do the selected models handle concurrency issues known from the traditional \ac{TL} approach?
\item How is the implementation efficiency of the selected models?
\end{enumerate}

Using the knowledge obtained from answering these questions, we will present an overview of the selected concurrency models. Specifically an overview of their characteristics and implementation efficiency, along with a set of suggestions for when to apply each model.

\subsection{Learning Goals}
Along with the definition of a problem statement, a set of learning goals have been defined. While answering the questions presented in the problem statement is the main focus, the learning goals can be viewed as a set of sub goals that are aimed at stimulating our learning.

The learning goals are defined as:
\begin{enumerate}
\item Learn what concurrency models exist and in what setting they are best applied.
\item Produce a sample implementation using each of the selected models in order to gain hands-on experience.
\item Get hands-on experience with the Scala\footnote{\url{http://www.scala-lang.org/}} programming language as it is gaining traction in the area of highly concurrent web services.\toby{Måske forklar før eller mere uddybnede forklaring af hvorfor netop Scala er interessant}
\end{enumerate}

\worksheetend