\makeatletter \@ifundefined{rootpath}{% Manual to memoir http://mirrors.dotsrc.org/ctan/macros/latex/contrib/memoir/memman.pdf

%\documentclass[a4paper,12pt,fleqn,openany,twoside]{memoir} %two sides for printing
\documentclass[a4paper,12pt,fleqn,openany,oneside]{memoir} %one side for pdf
\usepackage[english]{babel}
\usepackage[utf8]{inputenc}
\usepackage{microtype}
\usepackage{paralist}

%Definitions
\usepackage{amsthm}
\theoremstyle{plain}
\newtheorem{thm}{Theorem}[chapter] % reset theorem numbering for each chapter
\theoremstyle{definition}
\newtheorem{defn}[thm]{Definition}

% Choses the depth of numerations
\setsecnumdepth{subsection}

% Choses the depth of toc
%\maxtocdepth{subsection}

% LaTeX logical statements
\usepackage{ifthen}

% Fancy space after use of e.g. command
\usepackage{xspace}

% Skips after paragraphs
\usepackage{parskip}

% Layout settings
\setlength{\parindent}{0cm}
\setlength{\parskip}{2ex plus 2ex} %kan udvides til f.eks: '2ex plus 2ex minus 0ex'

\sloppybottom

% Don't make a collection per default
\newcommand{\worksheetcollection}{false}

% Bibtex
\usepackage[square,numbers,sort,comma]{natbib}
%\usepackage{cite}
%\bibliographystyle{plainnat}
\bibliographystyle{IEEEtran}


% Fixmes
\usepackage{fixme}
\fxsetup{draft}

% Mathematic
\usepackage{amsmath}
\usepackage{amsfonts}
\usepackage{amssymb}
\usepackage{stmaryrd}
\allowdisplaybreaks[1]


% Acronyms
\usepackage[printonlyused]{acronym}

% Images
\usepackage{graphicx}
\usepackage{wrapfig}
\usepackage[outdir=./]{epstopdf}
\usepackage{epsfig}


% Captions ans subcaptions
\captionnamefont{\footnotesize\bfseries}
\captiontitlefont{\footnotesize}

% Enable memoir subfloats for figures and tables
\newsubfloat{figure}
\newsubfloat{table}

% Hack memoir subfigure styles to have bold label and footnotesize fonts
\renewcommand{\thesubfigure}{\footnotesize\bfseries{(\alph{subfigure})}}
\renewcommand{\thesubtable}{\footnotesize\bfseries{(\alph{subtable})}}

\renewcommand{\subcaption}[2][]{\subbottom[\footnotesize{#1}]{#2}}

% Memoir tweak pagenumbers
%\pagestyle{headings}

% Tikz
\usepackage{tikz}
\usetikzlibrary{arrows,shapes,calc,positioning}
\pgfmathsetseed{1}

%Pgf plots
\usepackage{pgfplots}
\pgfplotsset{compat=1.5}
% loatbarrier, keep figures within (sub,subsub) sections
\usepackage{placeins}
\usepackage{pgfplots}
\usepgfplotslibrary{units}
\usepackage[space-before-unit,range-units = repeat]{siunitx}

% Hyperlinked auto references
\usepackage[hidelinks]{hyperref}
\usepackage[nameinlink]{cleveref}
\crefname{lstlisting}{Listing}{Listings}  
\Crefname{lstlisting}{Listing}{Listings}
%\def\chapterautorefname{Kapitel}
%\def\sectionautorefname{Afsnit}
%\def\subsectionautorefname{Afsnit}
%\def\subsubsectionautorefname{Underafsnit}
%\def\figureautorefname{Figur}
%\def\lstlistingautorefname{Listing}
%\def\lstnumberautorefname{Linje}
%\def\itemautorefname{Punkt}
\usepackage[hypcap]{caption} % Link to top of the figure and not the caption

%Sick shit to make \Autoref command
%http://tex.stackexchange.com/questions/36575/autorefs-inserted-text-has-not-the-correct-case
\def\HyLang@english{%
  \def\equationautorefname{Equation}%
  \def\footnoteautorefname{Footnote}%
  \def\itemautorefname{item}%
  \def\figureautorefname{Figure}%
  \def\tableautorefname{Table}%
  \def\partautorefname{Part}%
  \def\appendixautorefname{Appendix}%
  \def\chapterautorefname{Chapter}%
  \def\sectionautorefname{Section}%
  \def\subsectionautorefname{Subsection}%
  \def\subsubsectionautorefname{Subsubsection}%
  \def\paragraphautorefname{Paragraph}%
  \def\subparagraphautorefname{Subparagraph}%
  \def\FancyVerbLineautorefname{Line}%
  \def\theoremautorefname{Theorem}%
  \def\pageautorefname{Page}%
}

% Reference greencommentssections with number and name
\usepackage{nameref}
\newcommand{\bsnameref}[1]{\Cref{#1} ``\nameref{#1}''}
\newcommand{\bsref}[1]{\Cref{#1}}
\newcommand{\bsbilagref}[1]{Appendix \ref{#1}}
\newcommand{\bsbilagnameref}[1]{Appendix \ref{#1} ``\nameref{#1}''}
\newcommand{\pling}[1]{``#1''}


% Listings for code qoutes
\usepackage{listings}
%\usepackage[usenames,dvipsnames,svgnames,table]{xcolor}
\usepackage{color}
\usepackage{xcolor}
\definecolor{bluekeywords}{rgb}{0.13,0.13,1}
\definecolor{greencomments}{rgb}{0,0.5,0}
\definecolor{redstrings}{rgb}{0.9,0,0}
\usepackage{caption} 
\usepackage{multicol}
\DeclareCaptionFont{white}{\color{white}}
\DeclareCaptionFormat{listing}{\colorbox{gray}{\parbox{\textwidth}{#1#2#3}}}
\captionsetup[lstlisting]{format=listing,labelfont=white,textfont=white}
%\lstset{numbers=left}
\lstset{
	basicstyle=\footnotesize,
	tabsize=2,
	breaklines=true,
  literate={æ}{{\ae}}1 {ø}{{\o}}1 {å}{{\aa}}1 {Æ}{{\AE}}1 {Ø}{{\O}}1 {Å}{{\AA}}1,
  keywords={typeof, new, true, false, catch, function, return, null, catch, switch, var, if, in, while, do, else, case, break},
  keywordstyle=\color{blue}\bfseries,
  ndkeywords={class, export, boolean, throw, implements, import, this},
  ndkeywordstyle=\color{darkgray}\bfseries,
  identifierstyle=\color{black},
  sensitive=false,
  comment=[l]{//},
  morecomment=[s]{/*}{*/},
  commentstyle=\color{purple}\ttfamily,
  stringstyle=\color{red}\ttfamily,
  numbers=left,
  numbersep=-5pt,
  showstringspaces=false,
  showspaces=false,
  %morestring=[b]',
  %morestring=[b]"
}
\lstnewenvironment{code}[1][]%
  {\minipage{\linewidth} 
   \lstset{basicstyle=\ttfamily\footnotesize,frame=single,#1}}
  {\endminipage}


% Worksheet commands
\newcommand{\worksheetstart}[5]{ %Title, Revision, Date, Author, rootpath
	\ifthenelse{\equal{\worksheetcollection}{false}}{
		\newcommand{\rootpath}{#5}
		\documentheader
		\chapter{#1}
	}{
		\chapter{#1}
	}
	\vspace{-1em}
	\textbf{\tiny Revision #2 at #3. Written by #4}\\
	\vspace{2em}\\
}

\newcommand{\worksheetend}{
	\ifthenelse{\equal{\worksheetcollection}{false}}{
		\collectionend
	}{}
}

\newcommand{\documentheader}{
	% Draws a tikz camera
% #1 is the coordinate to the top left corner
% #2 is a label for the righthand center position
% #3 is the text shown in the center of the camera
\newcommand{\camera}[3]{
\coordinate (anchor) at #1;
\draw (anchor) -- ($ (anchor) + (0em,-20pt) $) -- ($ (anchor) + (10pt, -15pt) $) -- ($ (anchor) + (10pt,-5pt)$) -- cycle;
\draw ($ (anchor) + (10pt,-5pt) $) -- ($ (anchor) + (10pt,0pt) $) -- ($ (anchor) + (50pt,0pt) $) -- ($ (anchor) + (50pt,-20pt) $) -- node[yshift=10pt] {\tiny #3} ($ (anchor) + (10pt,-20pt) $)-- cycle;
\coordinate (#2) at ($ (anchor) + (50pt,-10pt) $);
}

\newcounter{frameNumber}
\newcommand{\frameWithSize}[3][false]{
	\stepcounter{frameNumber}
	\coordinate (anchor) at #2;
	\ifthenelse{\equal{#1}{false}}{
		\def\frameNumber{\arabic{frameNumber}}
	}{
		\def\frameNumber{#1}
	}
	\pgfmathtruncatemacro\randomnumber{random(0,4)}
	\node[yshift=20pt] at (anchor) {\frameNumber};
	\ifthenelse{\equal{#3}{I}}{
		\node[draw, minimum size=20pt, fill=green!60] at (anchor) {I};
		\filldraw[fill=gray] ($(anchor) + (-10pt,-40pt)$) rectangle ($(anchor) + (10pt,-20pt) + (0pt,\randomnumber pt)$);
	}{
		\ifthenelse{\equal{#3}{P}}{
			\node[draw, minimum size=20pt, fill=yellow!60] at (anchor) {P};
			\filldraw[fill=gray] ($(anchor) + (-10pt,-40pt)$) rectangle ($(anchor) + (10pt,-30pt) + (0pt,\randomnumber pt)$);
		}{
			\node[draw, minimum size=20pt, fill=blue!40!yellow!60!black] at (anchor) {\color{white}B};
			\filldraw[fill=gray] ($(anchor) + (-10pt,-40pt)$) rectangle ($(anchor) + (10pt,-37pt) + (0pt,\randomnumber pt)$);
		}
	}
	\draw[thick] ($(anchor) + (-10pt,-40pt)$) -- +(20pt,0pt);
}

	\begin{document}
	%\renewcommand{\chaptername}{Worksheet}
	\chapterstyle{section}
	\renewcommand{\beforechapskip}{0pt}
	\renewcommand{\afterchapskip}{0pt}
}

\newcommand{\collectionstart}[1]{
	\newcommand{\rootpath}{#1}
	\renewcommand{\worksheetcollection}{true}
	\documentheader
	\frontmatter
	%\forside
	\makeatletter \@ifundefined{rootpath}{% Manual to memoir http://mirrors.dotsrc.org/ctan/macros/latex/contrib/memoir/memman.pdf

%\documentclass[a4paper,12pt,fleqn,openany,twoside]{memoir} %two sides for printing
\documentclass[a4paper,12pt,fleqn,openany,oneside]{memoir} %one side for pdf
\usepackage[english]{babel}
\usepackage[utf8]{inputenc}
\usepackage{microtype}
\usepackage{paralist}

%Definitions
\usepackage{amsthm}
\theoremstyle{plain}
\newtheorem{thm}{Theorem}[chapter] % reset theorem numbering for each chapter
\theoremstyle{definition}
\newtheorem{defn}[thm]{Definition}

% Choses the depth of numerations
\setsecnumdepth{subsection}

% Choses the depth of toc
%\maxtocdepth{subsection}

% LaTeX logical statements
\usepackage{ifthen}

% Fancy space after use of e.g. command
\usepackage{xspace}

% Skips after paragraphs
\usepackage{parskip}

% Layout settings
\setlength{\parindent}{0cm}
\setlength{\parskip}{2ex plus 2ex} %kan udvides til f.eks: '2ex plus 2ex minus 0ex'

\sloppybottom

% Don't make a collection per default
\newcommand{\worksheetcollection}{false}

% Bibtex
\usepackage[square,numbers,sort,comma]{natbib}
%\usepackage{cite}
%\bibliographystyle{plainnat}
\bibliographystyle{IEEEtran}


% Fixmes
\usepackage{fixme}
\fxsetup{draft}

% Mathematic
\usepackage{amsmath}
\usepackage{amsfonts}
\usepackage{amssymb}
\usepackage{stmaryrd}
\allowdisplaybreaks[1]


% Acronyms
\usepackage[printonlyused]{acronym}

% Images
\usepackage{graphicx}
\usepackage{wrapfig}
\usepackage[outdir=./]{epstopdf}
\usepackage{epsfig}


% Captions ans subcaptions
\captionnamefont{\footnotesize\bfseries}
\captiontitlefont{\footnotesize}

% Enable memoir subfloats for figures and tables
\newsubfloat{figure}
\newsubfloat{table}

% Hack memoir subfigure styles to have bold label and footnotesize fonts
\renewcommand{\thesubfigure}{\footnotesize\bfseries{(\alph{subfigure})}}
\renewcommand{\thesubtable}{\footnotesize\bfseries{(\alph{subtable})}}

\renewcommand{\subcaption}[2][]{\subbottom[\footnotesize{#1}]{#2}}

% Memoir tweak pagenumbers
%\pagestyle{headings}

% Tikz
\usepackage{tikz}
\usetikzlibrary{arrows,shapes,calc,positioning}
\pgfmathsetseed{1}

%Pgf plots
\usepackage{pgfplots}
\pgfplotsset{compat=1.5}
% loatbarrier, keep figures within (sub,subsub) sections
\usepackage{placeins}
\usepackage{pgfplots}
\usepgfplotslibrary{units}
\usepackage[space-before-unit,range-units = repeat]{siunitx}

% Hyperlinked auto references
\usepackage[hidelinks]{hyperref}
\usepackage[nameinlink]{cleveref}
\crefname{lstlisting}{Listing}{Listings}  
\Crefname{lstlisting}{Listing}{Listings}
%\def\chapterautorefname{Kapitel}
%\def\sectionautorefname{Afsnit}
%\def\subsectionautorefname{Afsnit}
%\def\subsubsectionautorefname{Underafsnit}
%\def\figureautorefname{Figur}
%\def\lstlistingautorefname{Listing}
%\def\lstnumberautorefname{Linje}
%\def\itemautorefname{Punkt}
\usepackage[hypcap]{caption} % Link to top of the figure and not the caption

%Sick shit to make \Autoref command
%http://tex.stackexchange.com/questions/36575/autorefs-inserted-text-has-not-the-correct-case
\def\HyLang@english{%
  \def\equationautorefname{Equation}%
  \def\footnoteautorefname{Footnote}%
  \def\itemautorefname{item}%
  \def\figureautorefname{Figure}%
  \def\tableautorefname{Table}%
  \def\partautorefname{Part}%
  \def\appendixautorefname{Appendix}%
  \def\chapterautorefname{Chapter}%
  \def\sectionautorefname{Section}%
  \def\subsectionautorefname{Subsection}%
  \def\subsubsectionautorefname{Subsubsection}%
  \def\paragraphautorefname{Paragraph}%
  \def\subparagraphautorefname{Subparagraph}%
  \def\FancyVerbLineautorefname{Line}%
  \def\theoremautorefname{Theorem}%
  \def\pageautorefname{Page}%
}

% Reference greencommentssections with number and name
\usepackage{nameref}
\newcommand{\bsnameref}[1]{\Cref{#1} ``\nameref{#1}''}
\newcommand{\bsref}[1]{\Cref{#1}}
\newcommand{\bsbilagref}[1]{Appendix \ref{#1}}
\newcommand{\bsbilagnameref}[1]{Appendix \ref{#1} ``\nameref{#1}''}
\newcommand{\pling}[1]{``#1''}


% Listings for code qoutes
\usepackage{listings}
%\usepackage[usenames,dvipsnames,svgnames,table]{xcolor}
\usepackage{color}
\usepackage{xcolor}
\definecolor{bluekeywords}{rgb}{0.13,0.13,1}
\definecolor{greencomments}{rgb}{0,0.5,0}
\definecolor{redstrings}{rgb}{0.9,0,0}
\usepackage{caption} 
\usepackage{multicol}
\DeclareCaptionFont{white}{\color{white}}
\DeclareCaptionFormat{listing}{\colorbox{gray}{\parbox{\textwidth}{#1#2#3}}}
\captionsetup[lstlisting]{format=listing,labelfont=white,textfont=white}
%\lstset{numbers=left}
\lstset{
	basicstyle=\footnotesize,
	tabsize=2,
	breaklines=true,
  literate={æ}{{\ae}}1 {ø}{{\o}}1 {å}{{\aa}}1 {Æ}{{\AE}}1 {Ø}{{\O}}1 {Å}{{\AA}}1,
  keywords={typeof, new, true, false, catch, function, return, null, catch, switch, var, if, in, while, do, else, case, break},
  keywordstyle=\color{blue}\bfseries,
  ndkeywords={class, export, boolean, throw, implements, import, this},
  ndkeywordstyle=\color{darkgray}\bfseries,
  identifierstyle=\color{black},
  sensitive=false,
  comment=[l]{//},
  morecomment=[s]{/*}{*/},
  commentstyle=\color{purple}\ttfamily,
  stringstyle=\color{red}\ttfamily,
  numbers=left,
  numbersep=-5pt,
  showstringspaces=false,
  showspaces=false,
  %morestring=[b]',
  %morestring=[b]"
}
\lstnewenvironment{code}[1][]%
  {\minipage{\linewidth} 
   \lstset{basicstyle=\ttfamily\footnotesize,frame=single,#1}}
  {\endminipage}


% Worksheet commands
\newcommand{\worksheetstart}[5]{ %Title, Revision, Date, Author, rootpath
	\ifthenelse{\equal{\worksheetcollection}{false}}{
		\newcommand{\rootpath}{#5}
		\documentheader
		\chapter{#1}
	}{
		\chapter{#1}
	}
	\vspace{-1em}
	\textbf{\tiny Revision #2 at #3. Written by #4}\\
	\vspace{2em}\\
}

\newcommand{\worksheetend}{
	\ifthenelse{\equal{\worksheetcollection}{false}}{
		\collectionend
	}{}
}

\newcommand{\documentheader}{
	\input{\rootpath/setup/tikz-commands.tex}
	\begin{document}
	%\renewcommand{\chaptername}{Worksheet}
	\chapterstyle{section}
	\renewcommand{\beforechapskip}{0pt}
	\renewcommand{\afterchapskip}{0pt}
}

\newcommand{\collectionstart}[1]{
	\newcommand{\rootpath}{#1}
	\renewcommand{\worksheetcollection}{true}
	\documentheader
	\frontmatter
	%\forside
	\input{\rootpath/worksheets/titlepage/titlepage}
	\input{\rootpath/worksheets/preface/preface}
	%\input{\rootpath/worksheets/forord/forord}
	\newpage
	\newpage
	\tableofcontents*
	\mainmatter
}

\newcommand{\collectionend}{
	\backmatter
	\chapter{List of Acronyms}\vspace{3em}
	\input{\rootpath/setup/acronyms}
	\bibliography{\rootpath/setup/bibliography}
	\end{document}
}


\newcommand{\bscode}{
	\lstinline
}

\newcommand{\bscodemath}[1]{
	\text{\lstinline|#1|}
}

\newcommand{\bsqoute}[2]{
	\begin{quote}
		\textit{``#1''}
		\begin{center}
			-- \emph{#2}
		\end{center}
	\end{quote}
}


\newcommand{\lag}{\langle}
\newcommand{\rag}{\rangle}
\newcommand{\besk}[1]{\ensuremath{\lag #1 \rag}}

\newcommand{\namedtodo}[5]
{
  \ifthenelse{\equal{#1}{}}
  {
    \todo[color=#4,caption=
    {\textbf{#3: } #2}]
    {\color{#5}\textbf{#3: }#2}
  }
  {
    \todo[color=#4,caption=
    {\textbf{#3: } #1}
    ,inline]
    {\color{#5}\textbf{#3: }#2}
  }
}
\newcommand{\andreas}[2][]{\namedtodo{#1}{#2}{Andreas}{blue!50!red!10}{black}}
\newcommand{\lone}[2][]{\namedtodo{#1}{#2}{Lone}{orange}{black}}
\definecolor{babypink}{rgb}{0.96, 0.76, 0.76}
\newcommand{\toby}[2][]{\namedtodo{#1}{#2}{Tobias}{babypink}{black}}
\newcommand{\kasper}[2][]{\namedtodo{#1}{#2}{Kasper}{green}{black}}

%multicol
\usepackage{multicol}

% todonotes
%\usepackage[disable]{todonotes} %For final report
\usepackage{todonotes} %For writing notes
\usepackage{fancyvrb}

%Loading AAU macro
\usepackage{lastpage}
%%%%%%%%%%%%%%%%%%%%%%%%%%%%%%%%%%%%%%%%%%%%%%%%
% Macros for the titlepage
%%%%%%%%%%%%%%%%%%%%%%%%%%%%%%%%%%%%%%%%%%%%%%%%
%Creates the aau titlepage
\newcommand{\aautitlepage}[3]{%
  {
    %set up various length
    \ifx\titlepageleftcolumnwidth\undefined
      \newlength{\titlepageleftcolumnwidth}
      \newlength{\titlepagerightcolumnwidth}
    \fi
    \setlength{\titlepageleftcolumnwidth}{0.5\textwidth-\tabcolsep}
    \setlength{\titlepagerightcolumnwidth}{\textwidth-2\tabcolsep-\titlepageleftcolumnwidth}
    %create title page
    \thispagestyle{empty}
    \noindent%
    \begin{tabular}{@{}ll@{}}
      \parbox{\titlepageleftcolumnwidth}{
        \iflanguage{danish}{%
          \includegraphics[width=\titlepageleftcolumnwidth]{titlepage/figures/aau_logo_da}
        }{%
          \includegraphics[width=\titlepageleftcolumnwidth]{titlepage/figures/aau_logo_en}
        }
      } &
      \parbox{\titlepagerightcolumnwidth}{\raggedleft\small
        #2
      }\bigskip\\
       #1 &
      \parbox[t]{\titlepagerightcolumnwidth}{%
      \textbf{Abstract:}\bigskip\par
        \fbox{\parbox{\titlepagerightcolumnwidth-2\fboxsep-2\fboxrule}{%
          #3
        }}
      }\\
    \end{tabular}
    \vfill  
    \clearpage
  }
}

% Environment for problem statements
% Can be auto referenced.
\newtheorem{problem}{Problem}
\def\problemautorefname{Problem}

%Create english project info
\newcommand{\englishprojectinfo}[6]{%
  \parbox[t]{\titlepageleftcolumnwidth}{
    \textbf{Title:}\\ #1\bigskip\par
    %\textbf{Theme:}\\ #2\bigskip\par
    \textbf{Project Period:}\\ #2\bigskip\par
    \textbf{Project Group:}\\ #3\bigskip\par
    \textbf{Participants:}\\ #4\bigskip\par
    \textbf{Supervisor:}\\ #5\bigskip\par
    %\textbf{Copies:} #6\bigskip\par
    \textbf{Page Numbers:} \pageref{LastPage}\bigskip\par
    \textbf{Date of Completion:}\\ #6
  }
}



%Create danish project info
%\newcommand{\danishprojectinfo}[7]{%
 % \parbox[t]{\titlepageleftcolumnwidth}{
 %   \textbf{Titel:}\\ #1\bigskip\par
%    %\textbf{Tema:}\\ #2\bigskip\par
%    \textbf{Projektperiode:}\\ #2\bigskip\par
%    \textbf{Projektgruppe:}\\ #4\bigskip\par
%    \textbf{Deltager(e):}\\ #5\bigskip\par
 %   \textbf{Vejleder(e):}\\ #6\bigskip\par
%    \textbf{Oplagstal:} #7\bigskip\par
   % \textbf{Sidetal:} \pageref{LastPage}\bigskip\par
  %  \textbf{Afleveringsdato:}\\ #8
 % }
%}
}\makeatother
%\worksheetstart{Titlepage}{0}{December 31, 2012}{../../}
\begin{titlingpage}
\aautitlepage{%
  \englishprojectinfo{
    Investigation into hyped concurrency models %title
  }{%
    Fall Semester 2014 %project period
  }{%
    dpt907e14  % project group
  }{%
    %list of group members
    Tobias Ugleholdt Hansen\\
    Andreas Pørtner Karlsen\\ 
    Kasper Breinholt Laurberg\\
  }{%
    %list of supervisors
     Lone Leth Thomsen
  }{%
    \today % date of completion
  }%
}{%department and address
  \textbf{Department of Computer Science}\\
  Selma Lagerløfs Vej 300\\
  DK-9220 Aalborg Ø\\
  \href{http://www.cs.aau.dk}{http://www.cs.aau.dk}
}{% the abstract
Some nice abstract
}
\end{titlingpage}
	\makeatletter \@ifundefined{rootpath}{% Manual to memoir http://mirrors.dotsrc.org/ctan/macros/latex/contrib/memoir/memman.pdf

%\documentclass[a4paper,12pt,fleqn,openany,twoside]{memoir} %two sides for printing
\documentclass[a4paper,12pt,fleqn,openany,oneside]{memoir} %one side for pdf
\usepackage[english]{babel}
\usepackage[utf8]{inputenc}
\usepackage{microtype}
\usepackage{paralist}

%Definitions
\usepackage{amsthm}
\theoremstyle{plain}
\newtheorem{thm}{Theorem}[chapter] % reset theorem numbering for each chapter
\theoremstyle{definition}
\newtheorem{defn}[thm]{Definition}

% Choses the depth of numerations
\setsecnumdepth{subsection}

% Choses the depth of toc
%\maxtocdepth{subsection}

% LaTeX logical statements
\usepackage{ifthen}

% Fancy space after use of e.g. command
\usepackage{xspace}

% Skips after paragraphs
\usepackage{parskip}

% Layout settings
\setlength{\parindent}{0cm}
\setlength{\parskip}{2ex plus 2ex} %kan udvides til f.eks: '2ex plus 2ex minus 0ex'

\sloppybottom

% Don't make a collection per default
\newcommand{\worksheetcollection}{false}

% Bibtex
\usepackage[square,numbers,sort,comma]{natbib}
%\usepackage{cite}
%\bibliographystyle{plainnat}
\bibliographystyle{IEEEtran}


% Fixmes
\usepackage{fixme}
\fxsetup{draft}

% Mathematic
\usepackage{amsmath}
\usepackage{amsfonts}
\usepackage{amssymb}
\usepackage{stmaryrd}
\allowdisplaybreaks[1]


% Acronyms
\usepackage[printonlyused]{acronym}

% Images
\usepackage{graphicx}
\usepackage{wrapfig}
\usepackage[outdir=./]{epstopdf}
\usepackage{epsfig}


% Captions ans subcaptions
\captionnamefont{\footnotesize\bfseries}
\captiontitlefont{\footnotesize}

% Enable memoir subfloats for figures and tables
\newsubfloat{figure}
\newsubfloat{table}

% Hack memoir subfigure styles to have bold label and footnotesize fonts
\renewcommand{\thesubfigure}{\footnotesize\bfseries{(\alph{subfigure})}}
\renewcommand{\thesubtable}{\footnotesize\bfseries{(\alph{subtable})}}

\renewcommand{\subcaption}[2][]{\subbottom[\footnotesize{#1}]{#2}}

% Memoir tweak pagenumbers
%\pagestyle{headings}

% Tikz
\usepackage{tikz}
\usetikzlibrary{arrows,shapes,calc,positioning}
\pgfmathsetseed{1}

%Pgf plots
\usepackage{pgfplots}
\pgfplotsset{compat=1.5}
% loatbarrier, keep figures within (sub,subsub) sections
\usepackage{placeins}
\usepackage{pgfplots}
\usepgfplotslibrary{units}
\usepackage[space-before-unit,range-units = repeat]{siunitx}

% Hyperlinked auto references
\usepackage[hidelinks]{hyperref}
\usepackage[nameinlink]{cleveref}
\crefname{lstlisting}{Listing}{Listings}  
\Crefname{lstlisting}{Listing}{Listings}
%\def\chapterautorefname{Kapitel}
%\def\sectionautorefname{Afsnit}
%\def\subsectionautorefname{Afsnit}
%\def\subsubsectionautorefname{Underafsnit}
%\def\figureautorefname{Figur}
%\def\lstlistingautorefname{Listing}
%\def\lstnumberautorefname{Linje}
%\def\itemautorefname{Punkt}
\usepackage[hypcap]{caption} % Link to top of the figure and not the caption

%Sick shit to make \Autoref command
%http://tex.stackexchange.com/questions/36575/autorefs-inserted-text-has-not-the-correct-case
\def\HyLang@english{%
  \def\equationautorefname{Equation}%
  \def\footnoteautorefname{Footnote}%
  \def\itemautorefname{item}%
  \def\figureautorefname{Figure}%
  \def\tableautorefname{Table}%
  \def\partautorefname{Part}%
  \def\appendixautorefname{Appendix}%
  \def\chapterautorefname{Chapter}%
  \def\sectionautorefname{Section}%
  \def\subsectionautorefname{Subsection}%
  \def\subsubsectionautorefname{Subsubsection}%
  \def\paragraphautorefname{Paragraph}%
  \def\subparagraphautorefname{Subparagraph}%
  \def\FancyVerbLineautorefname{Line}%
  \def\theoremautorefname{Theorem}%
  \def\pageautorefname{Page}%
}

% Reference greencommentssections with number and name
\usepackage{nameref}
\newcommand{\bsnameref}[1]{\Cref{#1} ``\nameref{#1}''}
\newcommand{\bsref}[1]{\Cref{#1}}
\newcommand{\bsbilagref}[1]{Appendix \ref{#1}}
\newcommand{\bsbilagnameref}[1]{Appendix \ref{#1} ``\nameref{#1}''}
\newcommand{\pling}[1]{``#1''}


% Listings for code qoutes
\usepackage{listings}
%\usepackage[usenames,dvipsnames,svgnames,table]{xcolor}
\usepackage{color}
\usepackage{xcolor}
\definecolor{bluekeywords}{rgb}{0.13,0.13,1}
\definecolor{greencomments}{rgb}{0,0.5,0}
\definecolor{redstrings}{rgb}{0.9,0,0}
\usepackage{caption} 
\usepackage{multicol}
\DeclareCaptionFont{white}{\color{white}}
\DeclareCaptionFormat{listing}{\colorbox{gray}{\parbox{\textwidth}{#1#2#3}}}
\captionsetup[lstlisting]{format=listing,labelfont=white,textfont=white}
%\lstset{numbers=left}
\lstset{
	basicstyle=\footnotesize,
	tabsize=2,
	breaklines=true,
  literate={æ}{{\ae}}1 {ø}{{\o}}1 {å}{{\aa}}1 {Æ}{{\AE}}1 {Ø}{{\O}}1 {Å}{{\AA}}1,
  keywords={typeof, new, true, false, catch, function, return, null, catch, switch, var, if, in, while, do, else, case, break},
  keywordstyle=\color{blue}\bfseries,
  ndkeywords={class, export, boolean, throw, implements, import, this},
  ndkeywordstyle=\color{darkgray}\bfseries,
  identifierstyle=\color{black},
  sensitive=false,
  comment=[l]{//},
  morecomment=[s]{/*}{*/},
  commentstyle=\color{purple}\ttfamily,
  stringstyle=\color{red}\ttfamily,
  numbers=left,
  numbersep=-5pt,
  showstringspaces=false,
  showspaces=false,
  %morestring=[b]',
  %morestring=[b]"
}
\lstnewenvironment{code}[1][]%
  {\minipage{\linewidth} 
   \lstset{basicstyle=\ttfamily\footnotesize,frame=single,#1}}
  {\endminipage}


% Worksheet commands
\newcommand{\worksheetstart}[5]{ %Title, Revision, Date, Author, rootpath
	\ifthenelse{\equal{\worksheetcollection}{false}}{
		\newcommand{\rootpath}{#5}
		\documentheader
		\chapter{#1}
	}{
		\chapter{#1}
	}
	\vspace{-1em}
	\textbf{\tiny Revision #2 at #3. Written by #4}\\
	\vspace{2em}\\
}

\newcommand{\worksheetend}{
	\ifthenelse{\equal{\worksheetcollection}{false}}{
		\collectionend
	}{}
}

\newcommand{\documentheader}{
	\input{\rootpath/setup/tikz-commands.tex}
	\begin{document}
	%\renewcommand{\chaptername}{Worksheet}
	\chapterstyle{section}
	\renewcommand{\beforechapskip}{0pt}
	\renewcommand{\afterchapskip}{0pt}
}

\newcommand{\collectionstart}[1]{
	\newcommand{\rootpath}{#1}
	\renewcommand{\worksheetcollection}{true}
	\documentheader
	\frontmatter
	%\forside
	\input{\rootpath/worksheets/titlepage/titlepage}
	\input{\rootpath/worksheets/preface/preface}
	%\input{\rootpath/worksheets/forord/forord}
	\newpage
	\newpage
	\tableofcontents*
	\mainmatter
}

\newcommand{\collectionend}{
	\backmatter
	\chapter{List of Acronyms}\vspace{3em}
	\input{\rootpath/setup/acronyms}
	\bibliography{\rootpath/setup/bibliography}
	\end{document}
}


\newcommand{\bscode}{
	\lstinline
}

\newcommand{\bscodemath}[1]{
	\text{\lstinline|#1|}
}

\newcommand{\bsqoute}[2]{
	\begin{quote}
		\textit{``#1''}
		\begin{center}
			-- \emph{#2}
		\end{center}
	\end{quote}
}


\newcommand{\lag}{\langle}
\newcommand{\rag}{\rangle}
\newcommand{\besk}[1]{\ensuremath{\lag #1 \rag}}

\newcommand{\namedtodo}[5]
{
  \ifthenelse{\equal{#1}{}}
  {
    \todo[color=#4,caption=
    {\textbf{#3: } #2}]
    {\color{#5}\textbf{#3: }#2}
  }
  {
    \todo[color=#4,caption=
    {\textbf{#3: } #1}
    ,inline]
    {\color{#5}\textbf{#3: }#2}
  }
}
\newcommand{\andreas}[2][]{\namedtodo{#1}{#2}{Andreas}{blue!50!red!10}{black}}
\newcommand{\lone}[2][]{\namedtodo{#1}{#2}{Lone}{orange}{black}}
\definecolor{babypink}{rgb}{0.96, 0.76, 0.76}
\newcommand{\toby}[2][]{\namedtodo{#1}{#2}{Tobias}{babypink}{black}}
\newcommand{\kasper}[2][]{\namedtodo{#1}{#2}{Kasper}{green}{black}}

%multicol
\usepackage{multicol}

% todonotes
%\usepackage[disable]{todonotes} %For final report
\usepackage{todonotes} %For writing notes
\usepackage{fancyvrb}

%Loading AAU macro
\usepackage{lastpage}
%%%%%%%%%%%%%%%%%%%%%%%%%%%%%%%%%%%%%%%%%%%%%%%%
% Macros for the titlepage
%%%%%%%%%%%%%%%%%%%%%%%%%%%%%%%%%%%%%%%%%%%%%%%%
%Creates the aau titlepage
\newcommand{\aautitlepage}[3]{%
  {
    %set up various length
    \ifx\titlepageleftcolumnwidth\undefined
      \newlength{\titlepageleftcolumnwidth}
      \newlength{\titlepagerightcolumnwidth}
    \fi
    \setlength{\titlepageleftcolumnwidth}{0.5\textwidth-\tabcolsep}
    \setlength{\titlepagerightcolumnwidth}{\textwidth-2\tabcolsep-\titlepageleftcolumnwidth}
    %create title page
    \thispagestyle{empty}
    \noindent%
    \begin{tabular}{@{}ll@{}}
      \parbox{\titlepageleftcolumnwidth}{
        \iflanguage{danish}{%
          \includegraphics[width=\titlepageleftcolumnwidth]{titlepage/figures/aau_logo_da}
        }{%
          \includegraphics[width=\titlepageleftcolumnwidth]{titlepage/figures/aau_logo_en}
        }
      } &
      \parbox{\titlepagerightcolumnwidth}{\raggedleft\small
        #2
      }\bigskip\\
       #1 &
      \parbox[t]{\titlepagerightcolumnwidth}{%
      \textbf{Abstract:}\bigskip\par
        \fbox{\parbox{\titlepagerightcolumnwidth-2\fboxsep-2\fboxrule}{%
          #3
        }}
      }\\
    \end{tabular}
    \vfill  
    \clearpage
  }
}

% Environment for problem statements
% Can be auto referenced.
\newtheorem{problem}{Problem}
\def\problemautorefname{Problem}

%Create english project info
\newcommand{\englishprojectinfo}[6]{%
  \parbox[t]{\titlepageleftcolumnwidth}{
    \textbf{Title:}\\ #1\bigskip\par
    %\textbf{Theme:}\\ #2\bigskip\par
    \textbf{Project Period:}\\ #2\bigskip\par
    \textbf{Project Group:}\\ #3\bigskip\par
    \textbf{Participants:}\\ #4\bigskip\par
    \textbf{Supervisor:}\\ #5\bigskip\par
    %\textbf{Copies:} #6\bigskip\par
    \textbf{Page Numbers:} \pageref{LastPage}\bigskip\par
    \textbf{Date of Completion:}\\ #6
  }
}



%Create danish project info
%\newcommand{\danishprojectinfo}[7]{%
 % \parbox[t]{\titlepageleftcolumnwidth}{
 %   \textbf{Titel:}\\ #1\bigskip\par
%    %\textbf{Tema:}\\ #2\bigskip\par
%    \textbf{Projektperiode:}\\ #2\bigskip\par
%    \textbf{Projektgruppe:}\\ #4\bigskip\par
%    \textbf{Deltager(e):}\\ #5\bigskip\par
 %   \textbf{Vejleder(e):}\\ #6\bigskip\par
%    \textbf{Oplagstal:} #7\bigskip\par
   % \textbf{Sidetal:} \pageref{LastPage}\bigskip\par
  %  \textbf{Afleveringsdato:}\\ #8
 % }
%}
}\makeatother
\worksheetstart{Preface}{1}{December 17, 2013}{Mino}{../../}
This report documents project work done by group dpt907e14 at the Department of Computer Science at Aalborg University. The report was written as part of the Computer Science (IT) study program in the fall of 2014 at the 9th semester.

The first time an acronym is used it will appear in the format: Threads \& Locks (TL). Inline quotations and names will appear in \textit{italics}. The work presented in this report is based on work or results described in books, articles, video lectures and research papers from outside sources. The full list of acronyms along with the bibliography and appendix, can be found at the end of the report.

One chapter contains a number of mathematical formulas. The notation and intuition of these formulas will be defined after the formula. The exact understanding of the formulas is not important, it is the understanding of the amount of work the formulas will produce when calculated on a computer.

The report was written under the assumption, that the reader has solid understanding of computer science, and basic knowledge of concurrency. The basics of concurrency will be explained in \bsref{chp:con_basics}, and can be skipped if the reader already know this. The report can be read from end to end as a coherent piece, or as the reading guide below suggests.

Chapter 4, 5, and 6 can be read individually to obtain knowledge about a specific concurrency model. Chapter 7 compares the characteristics of the concurrency models and can be read to gain an overview of its characteristic differences. Chapter 8 can be read individually to see the performance result between the models. Chapter 9 can be used as a guide for choosing concurrency model, and contains only a summary of the most important factors.

We would like to give a special thanks to our supervisor Lone Leth Thomsen, from the Department of Computer Science, for excellent guidance, immaculate attention to detail. She supplied invaluable help throughout the project with professionalism and black humour. Her constructive criticism helped us to narrow down the subject and kept us motivated and enthusiastic about the project.


\newpage
\vspace*{30 mm}
%\vspace*{\fill}
\begin{vplace}

\begin{minipage}[b]{0.45\textwidth}
 \centering
 \rule{\textwidth}{0.5pt}\\
  Tobias Ugleholdt Hansen\\
 {\footnotesize tuha13@student.aau.dk}
\end{minipage}
\begin{minipage}[b]{0.45\textwidth}
 \centering
 \rule{\textwidth}{0.5pt}\\
  Andreas Pørtner Karlsen\\
 {\footnotesize akarls13@student.aau.dk}
\end{minipage}\\\\
\begin{minipage}[b]{0.45\textwidth}
 \centering
 \rule{\textwidth}{0.5pt}\\
  Kasper Breinholt Laurberg\\
 {\footnotesize klaurb13@student.aau.dk}
\end{minipage}\\\\


\end{vplace}
\worksheetend

	%\input{\rootpath/worksheets/forord/forord}
	\newpage
	\newpage
	\tableofcontents*
	\mainmatter
}

\newcommand{\collectionend}{
	\backmatter
	\chapter{List of Acronyms}\vspace{3em}
	\begin{acronym}[ITU-T]
\acro{WS}{Webservice}
\end{acronym}

	\bibliography{\rootpath/setup/bibliography}
	\end{document}
}


\newcommand{\bscode}{
	\lstinline
}

\newcommand{\bscodemath}[1]{
	\text{\lstinline|#1|}
}

\newcommand{\bsqoute}[2]{
	\begin{quote}
		\textit{``#1''}
		\begin{center}
			-- \emph{#2}
		\end{center}
	\end{quote}
}


\newcommand{\lag}{\langle}
\newcommand{\rag}{\rangle}
\newcommand{\besk}[1]{\ensuremath{\lag #1 \rag}}

\newcommand{\namedtodo}[5]
{
  \ifthenelse{\equal{#1}{}}
  {
    \todo[color=#4,caption=
    {\textbf{#3: } #2}]
    {\color{#5}\textbf{#3: }#2}
  }
  {
    \todo[color=#4,caption=
    {\textbf{#3: } #1}
    ,inline]
    {\color{#5}\textbf{#3: }#2}
  }
}
\newcommand{\andreas}[2][]{\namedtodo{#1}{#2}{Andreas}{blue!50!red!10}{black}}
\newcommand{\lone}[2][]{\namedtodo{#1}{#2}{Lone}{orange}{black}}
\definecolor{babypink}{rgb}{0.96, 0.76, 0.76}
\newcommand{\toby}[2][]{\namedtodo{#1}{#2}{Tobias}{babypink}{black}}
\newcommand{\kasper}[2][]{\namedtodo{#1}{#2}{Kasper}{green}{black}}

%multicol
\usepackage{multicol}

% todonotes
%\usepackage[disable]{todonotes} %For final report
\usepackage{todonotes} %For writing notes
\usepackage{fancyvrb}

%Loading AAU macro
\usepackage{lastpage}
%%%%%%%%%%%%%%%%%%%%%%%%%%%%%%%%%%%%%%%%%%%%%%%%
% Macros for the titlepage
%%%%%%%%%%%%%%%%%%%%%%%%%%%%%%%%%%%%%%%%%%%%%%%%
%Creates the aau titlepage
\newcommand{\aautitlepage}[3]{%
  {
    %set up various length
    \ifx\titlepageleftcolumnwidth\undefined
      \newlength{\titlepageleftcolumnwidth}
      \newlength{\titlepagerightcolumnwidth}
    \fi
    \setlength{\titlepageleftcolumnwidth}{0.5\textwidth-\tabcolsep}
    \setlength{\titlepagerightcolumnwidth}{\textwidth-2\tabcolsep-\titlepageleftcolumnwidth}
    %create title page
    \thispagestyle{empty}
    \noindent%
    \begin{tabular}{@{}ll@{}}
      \parbox{\titlepageleftcolumnwidth}{
        \iflanguage{danish}{%
          \includegraphics[width=\titlepageleftcolumnwidth]{titlepage/figures/aau_logo_da}
        }{%
          \includegraphics[width=\titlepageleftcolumnwidth]{titlepage/figures/aau_logo_en}
        }
      } &
      \parbox{\titlepagerightcolumnwidth}{\raggedleft\small
        #2
      }\bigskip\\
       #1 &
      \parbox[t]{\titlepagerightcolumnwidth}{%
      \textbf{Abstract:}\bigskip\par
        \fbox{\parbox{\titlepagerightcolumnwidth-2\fboxsep-2\fboxrule}{%
          #3
        }}
      }\\
    \end{tabular}
    \vfill  
    \clearpage
  }
}

% Environment for problem statements
% Can be auto referenced.
\newtheorem{problem}{Problem}
\def\problemautorefname{Problem}

%Create english project info
\newcommand{\englishprojectinfo}[6]{%
  \parbox[t]{\titlepageleftcolumnwidth}{
    \textbf{Title:}\\ #1\bigskip\par
    %\textbf{Theme:}\\ #2\bigskip\par
    \textbf{Project Period:}\\ #2\bigskip\par
    \textbf{Project Group:}\\ #3\bigskip\par
    \textbf{Participants:}\\ #4\bigskip\par
    \textbf{Supervisor:}\\ #5\bigskip\par
    %\textbf{Copies:} #6\bigskip\par
    \textbf{Page Numbers:} \pageref{LastPage}\bigskip\par
    \textbf{Date of Completion:}\\ #6
  }
}



%Create danish project info
%\newcommand{\danishprojectinfo}[7]{%
 % \parbox[t]{\titlepageleftcolumnwidth}{
 %   \textbf{Titel:}\\ #1\bigskip\par
%    %\textbf{Tema:}\\ #2\bigskip\par
%    \textbf{Projektperiode:}\\ #2\bigskip\par
%    \textbf{Projektgruppe:}\\ #4\bigskip\par
%    \textbf{Deltager(e):}\\ #5\bigskip\par
 %   \textbf{Vejleder(e):}\\ #6\bigskip\par
%    \textbf{Oplagstal:} #7\bigskip\par
   % \textbf{Sidetal:} \pageref{LastPage}\bigskip\par
  %  \textbf{Afleveringsdato:}\\ #8
 % }
%}
}\makeatother
\worksheetstart{Choosing Concurrency Model}{1}{April 24, 2013}{Kasper}{../../}
This chapter presents a recommendation for choosing the right concurrency model for a given problem. A single concurrency model is not a perfect solution for all problems, there are strengths and weaknesses for each model.\andreas{Siger hvem? Det gør vi, da vi har undersøgt det}\toby{Det gør JVM sykroniserings manden også} The programmer must identify and choose the model or models with the most strengths and least weaknesses. In each of the following sections we will provide some guidelines for when a particular concurrency model is best applied.

\label{chap:choosing_con_model}


\section{\acl{TL}}\label{sec:choose_TL}
The \ac{TL} concurrency model has been employed for concurrent implementations for many years and is a widely supported technique for concurrent programming. While the \ac{TL} concurrency model does provide good performance it has a number of issues with its usability. Examining its characteristics reveals a low level of abstraction, low simplicity, low readability and low writability. The number of concurrency related issues the programmer has to reason about when employing the model is a great contributor to these low characteristic scores.

The programmer must be adept in avoiding these issues in order to produce correct concurrent implementations. Failing to do so can lead to serious issues such as race conditions and deadlocks. In the context of fine grained locking, avoiding these issues becomes even harder. However, if the programmer has sufficient expertise in the area, fine grained locking can be employed to bolster performance.

While locking has been employed for many years to produce concurrent implementations its issues limits its practicality. Alternatives, attempting to deal with these issues, have started to gain traction. Potentially spurred on by the need for concurrency due to the changes in hardware, as described in \bsref{chap:intro}. The \ac{TL} concurrency models however still offers good performance for correct implementations. Furthermore the programmer is free to tune the performance via the selected lock granularity, provided the programmer has sufficient expertise to do so. If only simply synchronization is needed locking may be sufficient. Especially if the programmer has existing expertise in the area of lock based synchronization.

\kasper[inline]{Composability}
\kasper[inline]{Only TL available $=>$ vælg tl}
\toby[i]{Remember to look at other characteristics from study in concurrency, one being: implicit and explicit}

%If performance is critical
%Has a number of well known issues which require great effort to avoid
%Fine grained locking is difficult but provides good performance
%Requires experiensed programmers in order to produce good implementatons when employing fine grained locking
%Widely support in many languages
%Limited composablity
\section{\acl{STM}}\label{sec:choose_STM}
\ac{STM} has been an active area of research for almost 20 years\cite{shavit1997software}. It has been included in existing languages through both libraries\cite{rossbach2010transactional} and language changes\cite{harris2003language}, as well as included directly in newer languages such as Clojure. \ac{STM} eliminates the issue of deadlocks and scores around the middle or just above the middle in level of abstraction, simplicity, readability and writability. While \ac{STM} still require the programmer to explicitly apply synchronization to critical regions it manages to shift some of the workload away from the programmer. Memory transactions state that synchronization should be applied, but not how. The \ac{STM} system handles how synchronization is applied.

While memory transactions can be hard for novice programmers to understand, \ac{STM} offers a number of usability advantages compared to \ac{TL}. The elimination of deadlocks makes creating correct concurrent implementations with \ac{STM} simpler and composition trough nesting of transactions allows \ac{STM} based code segments to be combined without issues. 

With \ac{STM}, converting a sequential implementation to a thread safe one can be as simple as applying the atomic block to the needed code segments. The programmer needs not reason about deadlocks and the details of synchronization are handled by the \ac{STM} system. In more complex cases the existing sequential implementation may need to be split up to support concurrent work.

The availability of mature \ac{STM} implementations does however impact when the model can be applied. Some library based \ac{STM} implementations, such as those presented in \cite{ramadan2009committing} and \cite{herlihy2006flexible} require the use of \ac{STM} related types in order to track reads and writes. Language based implementations generally do no require such workarounds. The use of such types is undesirable as it may permeate parts of the source code and is far from the simplicity promised by the atomic block. Many programmers working in the industry will have requirements to which programming languages they employ. If these languages do not encompass a sufficiently simple \ac{STM} implementation, or maybe non at all, choosing another concurrency model may be desirable. If the problem at hand is simple, \ac{TL} is similar enough to \ac{STM} that the programmer may be able to employ that successfully instead.

As mentioned in \bsref{subsec:stm_irreversible_actions} STM has problems handling irreversible actions such as \ac{IO} and calls to native code. If the problem at hand encompasses a high degree of such operations then another concurrency may be preferred. If only a few of such operations are needed then another concurrency model can be employed for these code segments, while \ac{STM} can handle the remaining synchronization.

\ac{STM} must retry a memory transaction whenever a conflict occurs. In an scenario with many write operations and only a few read operations on the same data structure, the high number of conflicts may lead to reduced performance\cite[p. 311]{subramaniam2011programming}. However in a scenario consisting of only a few write operations and many read operations, many of the read operations can be executed concurrently.

%parallisation of existing code (in the clean format)
%Good alternative to locks, especially if performance is not the top priority and programmers are familar with the shared memory concurrency and critical regions
%Currently suffers from not being well integrated in many of the languages popular in the industry 
%Concept of transactions can be hard to understand for novices
%Problems with IO and native method calls
\kasper[inline]{Programming concurrency on the JVM. Side 186}
\section{The Actor Model}\label{sec:choose_Actor}
The actor model originated in 1973\cite[p. 14]{haller2012actors}. Most popular programming languages now have an actor implementation available\cite[p. 152]{sevenModels}, either directly in the language or indirectly through a library. One of the most popular actor based languages is Erlang. Erlang has built-in support for the actor models and has been a large factor in popularizing actor programming\cite[p. 152]{sevenModels}.

The actor model offers a simple concurrency model as it lets the programmer focus on high level abstractions and hides the details of synchronization and other low-level aspects, such as threads. Furthermore, adherence to the semantic properties of the actor model hides even fewer details e.g. with location transparency, the programmer does only need to know the unique name of an actor and not the physical location. However the programmer must be observant what properties the actor implementation he uses fulfils, as if it does not fulfils all, he has to manage the difficulties these properties ensures himself. Additionally, the actor model is very suitable for distributed programming, as a result of location transparency, that it avoids shared state and that it advocates asynchronous message passing.

In relation to our characteristics the actor model employs an explicit, fault restrictive and pessimistic model. Furthermore it has high simplicity, high orthogonality, slightly towards high readability, high level of abstraction, high expressivity and towards high writability. It is primarily the low level constructs and details that the actor model abstracts away, that results in these high characteristics. However, to abstract away these details the model is less expressive and optimistic, by not being able to use low-level threads and be able to choose which state variables to restrict access to, resulting in a restrictive and pessimistic model.

Programming with the actor model requires a different mindset as there is no shared state, which may be an initial difficulty for the programmer. Furthermore, it is still possible to produce deadlocks if one or more actors wait on one another for messages. Therefore it may not be suitable to use the actor model for an application that requires frequent interaction between actors where they have to wait on messages from one another. Additionally, there are some problems specific for actor programs, such as overflow of an actors mailbox, which the programmer now has to manage. 

All state variables being isolated as well as actors only handling a single message at a time, has the implication that multiple entities cannot read a state variable at the same time, which can result in lower concurrency. So if an application has frequent reads on some data structure, it may be beneficial to choose another model. Mixing the actor model with other languages constructs can help circumvent this limitation.% Although this should only be used for immutable variables as it will otherwise make race conditions possible. %også muligt at en actor implementation har en abstraction for det

The actor model offers a high level perspective on modelling concurrent implementations. The flow of such computations is however different from that of most programming languages. As such combining the actor model with elements of such languages can prove troublesome, especially if a sequential implementation already exists and is to be converted into a actor based implementation.\kasper{Mit bud op noget omkring actors og eksisterede kode.}

\toby[i]{Måske noget med at actor ikke fungere godt med eksisterende kode, som i andre snakkede om. Jeg kan dog ikke selv finde nogen ummidelbar begrundelse. Man kan jo godt kalde eksisterende kode inde fra actors. Og hvis det er i forhold til at der bruges tråde i den eksisterende kode, så er det vel lige så stort et problem med STM  der kalder eksisterende TL kode? Men jeg er meget åben overfor gode forklaringer af hvorfor det skulle være tilfældet at actor modellen er det :)}

%%%%EXTRA (Tror jeg har det meste med, men har ladet comments stå)

%semantic properties: atomic processing of messages, fairness, encapsulation and location transparency.

% Actors can be tested in isolation 

%which lets the programmer focus on creating application logic

%Distrubuted systems - nice fit
%Design of large systems (back ends)
%Allows desgin of systems on a more high level
%Problems moving from sequeltial to actor implementation
%Application build around actors
%Best for new implementations or reimplementation of existing systems
%Functions well with pipeline processsing of input
%Context swiching is cheap compared to the other models (thread ring)
 	% light weight actors (maybe refer til actor framework comparison article)
 	% Toby: Men dette er bare en implementerings strategi, vil vi snakke om den? Det er kun den implemetationer der har light-weight actrs det gælder for (men det er nok også de eneste som er widespread og anvendelige)
%Problems expressing work where actors are dependant on one another
%readability and writability impact on the model
	%high simplicity and orthogonality
	%high level of abstraction and expressivity
%performance results impact on the model

%Hewitt's team wanted a model that would not only simplify building such concurrent systems, but would also let them reason about concurrent programs in general
\subsection{Closing Thoughts}
Each of the selected concurrency models offers its own set of advantages. \ac{TL} can be fine tuned and is widely supported, \ac{STM} removes the issues of deadlock by employing memory transactions and the actor model offers a high level of abstraction as well as being applicable in a distributed setting. When choosing a concurrency model of a given implementation one must consider how the task at hand fits each of concurrency models strengths, and weaknesses.

As shown in the performance tests of \bsref{chap:performance} the chosen implementations of the selected concurrency models show little difference in performance during low synchronization scenarios. As the amount of synchronization increases the difference becomes slightly more prominent, giving the \ac{TL} concurrency model a slight lead in performance. However, if performance is not the key factor for a given implementation, the selection of concurrency model should be based on other characteristics that performance. %selecting a alternative, and less error prone, concurrency model could be of benefit. 

The \ac{STM} concurrency model is similar to the \ac{TL} concurrency model in that they both employ shared memory for communication and apply synchronization to critical regions. As such, if a programmer has existing knowledge the area of \ac{TL} concurrency, parts of that knowledge should be transferable to the area of \ac{STM} concurrency.

The actor model offers a more high level perspective on modelling concurrent implementations. By abstracting away many of the low level details the programmer can put additional effort into correctly expressing the concurrent implementation. Additionally the high level of abstraction allows the actor model to hide the distribution of actors, allowing uniform communication between both local and distributed actors.




\kasper[inline]{Combine models instead of competing}
\kasper[inline]{TL $>$ stm hvis der er meget IO}

\worksheetend